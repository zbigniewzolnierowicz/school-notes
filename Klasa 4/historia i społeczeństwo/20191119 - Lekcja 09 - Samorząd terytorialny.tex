\documentclass[a4paper]{article}
\usepackage[T1]{fontenc}
\usepackage[polish,english]{babel}
\usepackage[utf8]{inputenc}
\usepackage{lscape}
\usepackage{graphicx}
\usepackage{booktabs}
\usepackage{float}
\begin{document}
\title{{\huge Lekcja 08} \\
{\large Samorząd terytorialny}}
\author{Zbigniew Żołnierowicz}
\date{19.11.2019}
\maketitle
\section{Samorządy terytorialne}
\subsection{Gmina}
\begin{itemize}
    \item Władza wykonawcza (pochodzi z wyborów):
        \begin{itemize}
            \item w gminie wiejskiej - wójt
            \item w gminie miejsko-wiejskiej - burmistrz
            \item w gminie miejskiej - prezydent
        \end{itemize}
    \item Władza uchwałodawcza: rada gminy
\end{itemize}
\subsection{Powiat}
\begin{itemize}
    \item Władza wykonawcza: starosta powiatowy, wybrany przez zarząd powiatu
    \item Władza uchwałodawcza:
        \begin{itemize}
            \item rada powiatu, która wybiera spośród siebie
            \item zarząd powiatu
        \end{itemize}
\end{itemize}
\subsection{Województwo}
\begin{itemize}
    \item Władza wykonawcza: marszałek województwa, wybrany przez zarząd województwa
    \item Władza uchwałodawcza:
        \begin{itemize}
            \item sejmik
            \item wybrany z sejmika zarząd województwa
        \end{itemize}
\end{itemize}
\end{document}