\documentclass[a4paper]{article}
\usepackage[T1]{fontenc}
\usepackage[polish,english]{babel}
\usepackage{graphicx}
\usepackage[utf8]{inputenc}
\begin{document}
\title{{\huge Lekcja 01} \\
{\large Wymagania przedmiotowe}}
\author{Zbigniew Żołnierowicz}
\date{03.09.2019}
\maketitle
\section{Wymagania przedmiotowe}
Uczeń zobowiązany jest posiadać na lekcji materiały edukacyjne i notatki z poprzednich lekcji.
\subsection{Sprawdzanie wiedzy i ocenianie odbywa się na podstawie WSO, m.in.:}
    \begin{itemize}
        \item warunkiem zaliczenia semestru jest uzyskanie wymaganej liczby ocen \emph{obowiązkowych},
        napisanie zapowiedzianych kartkówek i uzyskanie średniej ocen na poziomie 1.67 i wyżej - czego życzę.
        \item w przypadku nieobecności ucznia na zapowiedzianej kartkówce umawia się on z nauczycielem na drugi termin,
        który musi być zrealizowany w ciągu 14 dni od pierwszego terminu pisania pracy
        \item w przypadku nieobecności ucznia na lekcji, podczas której wykonywana była karta pracy uczeń ma obowiązek
        nadrobić tę zaległość najpóźniej 14 dni przed końcem semestru.
    \end{itemize}
\subsection{W I Semestrze przewidywane są następujące oceny:}
    \begin{itemize}
        \item 2-3 - kartkówka (waga 3)
        \item 2-3 - karta pracy/ćwiczenia (waga 2)
        \item 1 - aktywność na lekcji/prezentacja (fakultatywnie) - waga 2
    \end{itemize}
\end{document}