\documentclass[a4paper]{article}
\usepackage[T1]{fontenc}
\usepackage[polish,english]{babel}
\usepackage[utf8]{inputenc}
\usepackage{lscape}
\usepackage{graphicx}
\usepackage{booktabs}
\usepackage{float}
\begin{document}
\title{{\huge Lekcja 06} \\
{\large Władza ustawodawcza w RP}}
\author{Zbigniew Żołnierowicz}
\date{21.10.2019}
\maketitle
\section{Organy sejmu}
\begin{description}
    \item[Prezydium sejmu] Marszałek i wicemarszałkowie. Decydują o sprawach zwią\-zanych z posiedzeniami Sejmu.
    Nigdzie nie jest napisane, ilu powinno być wicemarszałków, lecz zwykle, jest jeden wicemarszałek na klub parlamentarny.
    \item[Konwent seniorów] Prezydium sejmu i szefowie klubów parlamentarnych.
    \item[Komisje sejmowe] Powoływane do: \begin{itemize}
        \item rozpatrywania i przygotowywania spraw stanowiących przedmiot\\prac Sejmu,
        \item wyrażania opinii w sprawach przekazanych pod ich obrady przez Sejm, Marszałka Sejmu lub Prezydium Sejmu
        \item kontroli sejmowej w zakresie określonym Konstytucją i ustawami.
    \end{itemize}
    \item[Komisje śledcze] Tylko sejm może je utworzyć
\end{description}
\section{Funkcje sejmu}
\begin{description}
    \item[Funkcja ustrojodawcza] nadanie ustroju (zmiana Konstytucji, etc.)
    \item[Funkcja ustawodawcza] stanowienie prawa (uchwalanie ustaw)
    \item[Funkcja kreacyjna] powoływanie osób na najważniejsze stanowiska w państwie, powołanie rządu
    \item[Funkcja legitymizacyjna] uprawomocnianie ustroju państwa
    \item[Funkcja kontrolna] nadzorowanie pracy rządu
    \item[Funkcja sądownicza] osądzanie osób zajmujących najwyższe stanowiska w\\państwie
    \item[Absolutorium] wyrażenie zgody przez Sejm na ustawę budżetową
    \item[Zapytania i interpelacje poselskie]  
\end{description}
\end{document}
