\documentclass[a4paper]{article}
\usepackage[T1]{fontenc}
\usepackage[polish,english]{babel}
\usepackage[utf8]{inputenc}
\usepackage{lscape}
\usepackage{graphicx}
\usepackage{booktabs}
\usepackage{float}
\begin{document}
\title{{\huge Lekcja 05} \\
{\large Demokracja: rządy ludu czy prawa?}}
\author{Zbigniew Żołnierowicz}
\date{30.09.2019}
\maketitle
\begin{description}
    \item[Zasada suwerenności narodu] Decyzja o tym, kto rządzi w kraju jest decyzją ostateczną narodu
    \item[Pluralizm] mnogość partii politycznych
    \item[Zasada praworządności] wszyscy poruszają się w obrębie prawa
    \item[Konstytucjonalizm] Konstytucja jest najwyższym aktem prawnym, którego nie można podważyć inną ustawą
    \item Podział i równowaga władz
    \item Podejmowanie decyzji większością przy poszanowaniu praw mniejszości
\end{description}
\end{document}