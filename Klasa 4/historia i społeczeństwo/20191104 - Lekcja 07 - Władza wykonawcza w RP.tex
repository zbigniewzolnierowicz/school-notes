\documentclass[a4paper]{article}
\usepackage[T1]{fontenc}
\usepackage[polish,english]{babel}
\usepackage[utf8]{inputenc}
\usepackage{lscape}
\usepackage{graphicx}
\usepackage{booktabs}
\usepackage{float}
\begin{document}
\title{{\huge Lekcja 07} \\
{\large Władza wykonawcza w RP}}
\author{Zbigniew Żołnierowicz}
\date{04.11.2019}
\maketitle
\section{Prezydent}
\begin{itemize}
    \item Mała faktyczna moc
    \item Wybierany na kadencję 5-letnią
    \item Może rządzić max. 2 kadencje
    \item Może cofnąć ustawę do trybunału konstytucyjnego
    \item Może zawetować jakąkolwiek ustawę
    \item Wybierany przez Naród w wyborach powszechnych, równych, bezpośrednich i w głosowaniu tajnym
    \item Obywatel polski, który najpóźniej w dniu wyborów kończy 35 lat i korzysta z pełni praw wyborczych do Sejmu. Kandydata zgłasza conajmniej 100000 obywateli mających prawo wybierania do sejmu
    \item Wybierany zostaje kandydat, który otryma więcej niż połowę głosów. Jeżeli żaden z kandydatów nie otrzyma większości, robiona jest druga tura wyborów
\end{itemize}
\begin{description}
    \item[Kontrasygnata] Akty urzędowe prezydenta nie będą ważne bez podpisu (kontrasygnaty) Premiera.
\end{description}
\section{Rada ministrów}
\end{document}