\documentclass[a4paper]{article}
\usepackage[T1]{fontenc}
\usepackage[polish,english]{babel}
\usepackage[utf8]{inputenc}
\usepackage{lscape}
\usepackage{graphicx}
\usepackage{booktabs}
\usepackage{float}
\begin{document}
\title{{\huge Lekcja 04} \\
{\large Na politycznej scenie}}
\author{Zbigniew Żołnierowicz}
\date{23.09.2019}
\maketitle
\section{Definicja partii politycznej i proces założenia partii}
\subsection{Co trzeba zrobić żeby założyć partię?}
\begin{itemize}
    \item Do założenia partii trzeba mieć 18 lat.
    \item Partia musi posiadać unikalną nazwę i unikalne logo.
    \item Partia musi zebrać 1000 podpisów.
    \item Partia musi ustanowić statut partii wyszczególniający prawa i obowiązki jej członków oraz określający zasady jej działania.
    \item Partia nie może odwoływać się do praktyk i metod ideologii uznanych za totalitarne.
    \item Partia musi być zarejestrowana w warszawskim Sądzie Okręgowym.
\end{itemize}
\end{document}