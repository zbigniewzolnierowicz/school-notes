\documentclass[a4paper]{article}
\usepackage[T1]{fontenc}
\usepackage[polish,english]{babel}
\usepackage[utf8]{inputenc}
\usepackage{lscape}
\usepackage{graphicx}
\usepackage{booktabs}
\usepackage{float}
\begin{document}
\title{{\huge Lekcja 08} \\
{\large Władza sądownicza w RP}}
\author{Zbigniew Żołnierowicz}
\date{12.11.2019}
\maketitle
\section{Zasady wymiaru sprawiedliwości}
\section{Krajowa rada sądownicza}
\begin{itemize}
    \item Stoi na straży niezależności i niezawisłości sądów
    \item Istnieje od 1989
    \item Członkowie: \begin{itemize}
        \item Pierwszego Prezesa Sądu Najwyższego, Prezesa Naczelnego Sądu Administracyjnego oraz ministra sprawiedliwości,
        \item jednej osoby powołanej przez Prezydenta RP,
        \item czterech posłów na Sejm wybranych przez Sejm oraz dwóch senatorów wybranych przez Senat,
        \item piętnastu sędziów wybranych przez Sejm spośród: sędziów Sądu Najwyższego, sądów powszechnych, administracyjnych i wojskowych, z uwzględnieniem (w miarę możliwości) reprezentacji sędziów różnych rodzajów i szczebli sądów.
    \end{itemize}
    \item Kadencje \begin{itemize}
        \item Wybieranych członków: 4 lata
        \item Pozostałych: powiązana z kadencją ich stanowisk (minister spra\-wiedliwości, Pierwszy Prezes SN i Prezes NSA) lub nieokreślona (osoba powołana przez prezydenta)
    \end{itemize}
    \item Siedziba: dawny budynek Szkoły Głównej Gospodarstwa Wiejskiego przy ul. Rakowieckiej 26/30
\end{itemize}
\end{document}