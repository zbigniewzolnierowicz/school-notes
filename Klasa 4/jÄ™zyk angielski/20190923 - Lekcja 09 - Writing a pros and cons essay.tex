\documentclass[a4paper]{article}
\usepackage[T1]{fontenc}
\usepackage[spanish,english,polish]{babel}
\usepackage{graphicx}
\usepackage{booktabs}
\usepackage[normalem]{ulem}
\usepackage{float}
\usepackage{indentfirst}
\usepackage{soul}
\usepackage[utf8]{inputenc}
\useunder{\uline}{\ul}{}
\begin{document}
\title{{\huge Lekcja 08} \\
{\large Writing a ``pros'' and ``cons'' essay}}
\author{Zbigniew Żołnierowicz}
\date{23.09.2019}
\maketitle
\section*{Rozprawka}
\begin{quote}
    Obecnie, w czasach masowej turystyki europejskie miasta popularne wśród turystów dążą różnymi sposobami do ograniczenia napływu zwiedzających. Napisz rozprawkę wypisującą zalety i wady takiego rozwiązania.
\end{quote}
\subsection*{Wstęp}
In the age of mass tourism, many big and historically relevant cities in Europe are starting to limit the inflow of tourists into their city.
It is manifested by many different strategies utilized by the local governments to drive the tourists out of the city.
\subsection*{Rozwinięcie}
Such a tactic is obviously justified, because, from the point of view of the average citizen, the visitors are loud, leave their garbage everywhere and because of them, the prices in the shops have gone up. Getting rid of tourists would be a benefit to the common man - but not necessarily to the city.

Nevertheless, other than trash, tourists also bring a ton of money to the local economy. They go to gift shops to purchase souvenirs, eat at local restaurants and, if we're dealing with foreigners, exchange money in their currency exchanges, and the more money they use, the more money the citizens get, and thus the government gets more taxes.

Furthermore, the city benefits from the tourists' social media posts, because they act as basically free advertising for the city.
\subsection*{Zakończenie}
Of course, getting rid of tourists and tourism outright isn't the correct answer, and should be considered on a city to city basis. There is no ``right'' answer to this problem, because the solution lays within the politics of the local region (and sometimes politics of the country) - should the government prioritize the well-being of citizens and cleanliness of cities, or, controversial though it may seem to be, forgo the former and maximize the amount of income coming into the ``president's coffers''?
\end{document}