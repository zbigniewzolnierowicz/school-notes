\documentclass{article}
\usepackage[T1]{fontenc}
\usepackage[spanish,english,polish]{babel}
\usepackage{graphicx}
\usepackage[utf8]{inputenc}
\begin{document}
\title{{\huge Lekcja 24} \\
{\large Advanced vocabulary practice}}
\author{Zbigniew Żołnierowicz}
\date{03.09.2019}
\maketitle
\section{Sprawdziany}
Każdy uczeń musi podejść do wszystkich sprawdzianów. Wszystkie sprawdziany zapowiedziane, dwa tygodnie na poprawę
\subsection{Ilość sprawdzianów w pierwszym semestrze}
2 sprawdziany i matura próbna
\subsection{Ilość sprawdzianów w drugim semestrze}
3 sprawdziany
\section{Kartkówki}
Ilość kartkówek na semestr: 4 - 5.
\section{Podręcznik, zeszyt ćwiczeń, odpowiedź ustna i nieprzygotowanie}
\subsection{Podręcznik i zeszyt ćwiczeń}
Matura Focus 5
\subsection{Nieprzygotowanie}
2 na semestr
\section{Praca na lekcji}
\subsection{Plusy}
Za aktywność plusy, 5 plusów = ocena bardzo dobra.
\subsection{Minusy}
Za rażące wyłączenie się z toku lekcji uczeń otrzymuje ocenę niedostateczną.
\section{Ocenianie}
Procentowe progi zaliczenia form sprawdzania wiedzy uzależnione są od terminów
przystąpienia do ich napisania. Wyższe progi procentowe stosuje się wówczas,
gdy uczeń nie stawi się na sprawdzian w pierwszym terminie bez podania uzasadnionej przyczyny.
\begin{table}[]
    \centering
    \resizebox{\textwidth}{!}{%
    \begin{tabular}{|l|l|l|}
    \hline
    Stopień & I termin {[}\% punktów{]} & II termin    \\ \hline
    cel     & 100                       & 100          \\ \hline
    bdb     & 91 - 99                   & 96 - 99      \\ \hline
    db      & 71 - 90                   & 81 - 95      \\ \hline
    dst     & 51 - 70                   & 61 - 80      \\ \hline
    dop     & 41 - 50                   & 51 - 60      \\ \hline
    ndst    & 40 i poniżej              & 50 i poniżej \\ \hline
    \end{tabular}%
    }
    \end{table}
\begin{description}
    \item[sprawdzian] 6
    \item[poprawa sprawdzianu] 6
    \item[próbna matura] 5
    \item[kartkówka] 3
    \item[odp. ust.] 3
    \item[projekt] 3 do 5
    \item[zad. dom.] 2
    \item[aktywność] 1
\end{description}
\end{document}