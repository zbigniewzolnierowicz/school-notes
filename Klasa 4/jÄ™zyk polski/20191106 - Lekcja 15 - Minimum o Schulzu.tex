\documentclass[a4paper]{article}
\usepackage[T1]{fontenc}
\usepackage[polish,english]{babel}
\usepackage[utf8]{inputenc}
\usepackage{lscape}
\usepackage{graphicx}
\usepackage{booktabs}
\usepackage{float}
\usepackage{dirtytalk}
\begin{document}
\title{{\huge Lekcja 15} \\
{\large Minimum o Schulzu}}
\author{Zbigniew Żołnierowicz}
\date{06.11.2019}
\maketitle
\section{Stałe motywy w twórczości Schulza}
\subsection{Grafiki - Xięga Bałwochwalcza}
\subsection{Słowa-klucze w twórczości Schulza}
\begin{description}
    \item[księga] 
    \item[księga wakacji]
    \item[dzieciństwo]
    \item[inicjacja]
    \item[metamorfoza]
    \item[wyobraźnia]
    \item[tajemnica]
    \item[dziwność]
    \item[groteska] dziwność, elementy karykatury, deformacja postaci ludzkich
    \item[kreacja]
    \item[animizacja] człowiek przedstawiony jako zwierze
    \item[personifikacja] konie przedstawione jako mężczyźni
    \item[erotyka]
    \item[zmysłowość świata]
    \item[mit]
    \item[\emph{mityzacja rzeczywistości}]
    \item[archetyp/prawzorzec]
    \item[świat/kosmos]
    \item[oniryzm/poetyka snu]
    \item[czasoprzestrzeń] 
    \item[świat przedstawiony] jest antymimetyczny 
\end{description}
\end{document}