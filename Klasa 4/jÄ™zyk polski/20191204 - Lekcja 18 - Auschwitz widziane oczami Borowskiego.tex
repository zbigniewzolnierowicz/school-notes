\documentclass[a4paper]{article}
\usepackage[T1]{fontenc}
\usepackage[polish,english]{babel}
\usepackage[utf8]{inputenc}
\usepackage{lscape}
\usepackage{graphicx}
\usepackage{booktabs}
\usepackage{float}
\usepackage{dirtytalk}
\begin{document}
\title{{\huge Lekcja 18} \\
{\large Auschwitz widziane oczami Borowskiego}}
\author{Zbigniew Żołnierowicz}
\date{04.12.2019}
\maketitle
\section{Pytania}
\subsection{Kim jest narrator?}
Narratorem jest Tadeusz, który trafił do obozu Birkenau i dostał szkolenie na sanitariusza
\subsection{Jaki ma stosunek wobec okazywanych zdarzeń?}
Przyzwyczajenie, rezygnacja, próba normalnego życia. Czuje zdenerwowanie wobec osób, które próbują łudzić się, że ich życie w obozie śmierci.
\subsection{W jaki sposób opisuje innych więźniów?}
Jako równych sobie, a niektórych (szczególnie tych z wyższymi numerami) jako lepszych od siebie.
\section{Definicje}
\begin{description}
    \item[Narracja behawiorystyczna] Tadeusz Borowski opisuje innych więźniów na podstawie ich zachowań (brak motywów filozoficznych innych postaci)
    \item[Teheroizacja] Szara rzeczywistość, nie ma wielkiego bohatera.
\end{description}
\section{Nieludzkie warunki życia}
\begin{itemize}
    \item ciężka praca
    \item racje żywnościowe ledwo żeby przeżyć
    \item wysoka gęstość ludzi
    \item tyraniczna biurokratyzacja śmierci
    \item skrawki normalności na tle mordu, eksperymentów i niewolnictwa
    \item przyzwyczajenie ludzi do ich sytuacji
\end{itemize}
\end{document}
