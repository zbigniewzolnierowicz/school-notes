\documentclass[a4paper]{article}
\usepackage[T1]{fontenc}
\usepackage[polish,english]{babel}
\usepackage[utf8]{inputenc}
\usepackage{lscape}
\usepackage{graphicx}
\usepackage{booktabs}
\usepackage{float}
\usepackage{dirtytalk}
\begin{document}
\title{{\huge Lekcja 10} \\
{\large Minimum o Gombrowiczu!}}
\author{Zbigniew Żołnierowicz}
\date{16.10.2019}
\maketitle
\section{Tytuł: ``Ferdydurke''}
Słowo ``Ferdydurke'' nic nie znaczy, literacka prowokacja: zaciekawia, intryguje, zapowiada niemimetyczną, nierealistyczną, nieprawdopodobną rzeczywistość \\powieści.
\section{Funkcja zakończenia}
\begin{quotation}
    Koniec i bomba.
    A kto czytał ten trąba!
\end{quotation}
\section{Problem: charakter więzi międzyludzkich}
Gombrowicz udowadnia, że naturalne odruchy zostały stłumione:
\begin{quotation}
    \say{W człowieku nie ma nic oryginalnego i naturalnego, jesteśmy tylko formą.}
\end{quotation}
\begin{quotation}
    \say{Przekleństwem każdego człowieka jest drugi człowiek.}
\end{quotation}
\section{Filozofia - Model człowieka wg. Gombrowicza}
Ludzkie życie to tylko zamian kolejnych masek, pod którymi nie ma nic. Dla Gombrowicza dzieciństwo jest przekleństwem człowieka.
\begin{quotation}
    \say{Każdy człowiek jest dzieckiem podszyty.}
\end{quotation}
\section{Forma}
\begin{description}
    \item[Pupa (szkoła)] niedojrzałość
    \item[Upupić] proces zdziecinnienia, niesamodzielnego myślenia, podporządkowania innym
    \item[Łydka (młodziakowie, pensja)] nowoczesność, niedojrzałość, erotyczne po\-rządanie
    \item[Gęba] maska, rola
    \item[Przyprawić komuś gębę] przedstawić kogoś w krzywym zwierciadle oszczerstwa
    \item \say{Nie ma uczieczki przed gębą, jak tylko w inną gębę}
    \item[Parobek (ziemiański dworek Hurleckich)] Symbol zacofania, natury czystej, ludzkiej, własnej twarzy, prostactwo
\end{description}
\section{Język}
\begin{quotation}
    \say{Bełkot na miarę giewontu} --- Leopold Tyrmand
\end{quotation}
\begin{quotation}
    \say{Syfona zgwałcić muszę przez uszy!}
\end{quotation}
\begin{quotation}
    \say{Nogi miał na czole!}
\end{quotation}
\begin{quotation}
    \say{Daj gębą swą komuś po zębach!}
\end{quotation}
\section{Gatunek}
\subsection{Sunkretyzm gatunkowy}
Powiastka filozoficzna, gawęda, powieść o dojrzewaniu, pamiętnik, powieść przygodowa
\subsection{Brak porządku logicznego}
Pogranicze jawy i snu - oniryzm
\subsection{Brak związków przyczynowo-skutkowych}
\subsection{Brak psychologii postaci opartej na zasadach życio\-wego prawdopodobieństwa}
\subsection{Gra z czytelnikiem}
\subsection{Gry językowe}
\subsection{Narrator}
\subsubsection{Cechy autora - Gombrowicza}
\subsubsection{Cechy bohatera - Józia}
\subsection{Kompozycja otwarta (historia Józia?)}
\section{Przydatne terminy}
\begin{itemize}
    \item Groteska
    \item Parodia (pojedynku, języka romantyków, lekcji)
\end{itemize}
\appendix {\bf Zadanie domowe:} Jak nauczyciele chcą nas upupić? (opowiadanie,\\ artykuł, dramat)
\begin{quotation}
Gdy belfer do matury zagania \\
Z szarej materii robi się kania \\
Podczas gdy każdy wie, \\
Że po maturze się \\
Wie tyle o temacie co podczas spania. \\
Matymatyka przedmiot przydatny, \\
Lecz polski jest trochę odczapny. \\
Gdy jesteś informatykiem \\
Nikt nosa ci prztykiem \\
Nie wytknie że zasób twój wiedzy nieadekwatny. \\
Podsumowując tę farsę, \\
Jak chcesz podcierać się hajsem \\
Nie idź na uniwerek, \\
Ale włóż parę szelek \\
I rozwijaj sam swoje pasje.
\end{quotation}
\end{document}