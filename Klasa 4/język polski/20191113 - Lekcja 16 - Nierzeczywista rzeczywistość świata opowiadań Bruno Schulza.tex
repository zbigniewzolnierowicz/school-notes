\documentclass[a4paper]{article}
\usepackage[T1]{fontenc}
\usepackage[polish,english]{babel}
\usepackage[utf8]{inputenc}
\usepackage{lscape}
\usepackage{graphicx}
\usepackage{booktabs}
\usepackage{float}
\usepackage{dirtytalk}
\begin{document}
\title{{\huge Lekcja 16} \\
{\large Nierzeczywista rzeczywistość świata opowiadań Bruno Schulza}}
\author{Zbigniew Żołnierowicz}
\date{13.11.2019}
\maketitle
\section{Plan akcji}
\begin{enumerate}
    \item Ojcowskie brednie
    \item Wyjście do teatru
    \item Zapomniany portfel
    \item Opis sklepu cynamonowego
    \item Tylna ściana gimnazjum
    \item Wspomnienia lekcji rysunku
    \item Oddanie dorożki bohaterowi
    \item Podróż
    \item Odkrycie rany i porzucenie konia
    \item Spacer z kolegami ze szkoły i magia nocy
\end{enumerate}
\section{Karta pracy}
\subsection{Kim jest narrator opowiadania \emph{Sklepy cynamonowe}? Uzasadnij swoją odpowiedź.}
Narratorem opowiadania jest chłopiec w wieku gimnazjalnym lub ponadgimnazjalnum, co możemy wywnioskować z tego,
że pod koniec opowiadania spotyka się ze znajomymi z klasy i wspomina o zajęciach z rysunku w gimnazjum. Ma on ojca i matkę.
\\ Narrator jest subiektywny lub pamiętnikarski.
\pagebreak
\subsection{Przeczytaj przytoczony fragment opowiadania i prze\-analizuj zastosowane w nim środki stylistyczne.\\Wytłumacz, jaką pełnią funkcję w opisie i interpretacji świa\-ta.}
\begin{quotation}
    W okresie najkrótszych, sennych dni zimowych, ujętych z obu stron, od poranku i od wieczora, w futrzane krawędzie zmierzchów, gdy miasto rozgałęziało się coraz głębiej w labirynty zimowych nocy, z trudem przywoływane przez krótki świt do opamiętania, do\\powrotu[\dots]
\end{quotation}
\begin{table}[H]
    \resizebox{\textwidth}{!}{
        \begin{tabular}{|l|l|l|}
        \hline
        \multicolumn{1}{|c|}{\textbf{Cytat}} & \multicolumn{1}{c|}{\textbf{Nazwa środka stylistycznego}} & \multicolumn{1}{c|}{\textbf{Funkcja}} \\ \hline
        \begin{tabular}[c]{@{}l@{}}``gdy miasto[\dots] z trudem przywoływane\\ do opamiętania''\end{tabular} & \begin{tabular}[c]{@{}l@{}}personifikacja,\\ onomatopeja\end{tabular} & Nadanie miastu osobowości \\ \hline
        \begin{tabular}[c]{@{}l@{}}``[\dots]labirynty zimowych nocy,[\dots]''\\ ``[\dots]futrzane krawędzie zmierzchów[\dots]''\end{tabular} & metafora & \begin{tabular}[c]{@{}l@{}}Wprowadzenie poczucia tajemnicy i\\ charakteryzacja nocy\end{tabular} \\ \hline
        ``senne dni zimowe'' & animizacja & przywołanie rzeczy do życia \\ \hline
        ``[\dots]futrzane krawędzie zmierzchów[\dots]'' & synestezja & działanie na zmysły \\ \hline
        \end{tabular}
    }
\end{table}
\subsection{Wyjaśnij pojęcie \emph{oniryzm} i określ jego związek ze\\\emph{Sklepami cynamonowymi}}
Oniryzm to konwencja literacka, ukazująca rzeczywistość z pomieszaniem snu i jawy. W \emph{Sklepach cynamonowych} jest to ukazane m.in.: w postaci tytułowych sklepach cynamonowych, w których sprzedawane są niesamowite, egzotyczne i nierealne przedmioty, jak homunkulusy.
\subsection{W konteście teorii Zygmunta Freuda dotyczącej snu zinterpretuj podany fragment opowiadania Schulza.}
Główny bohater \emph{Sklepów Cynamonowych} podświadomie pragnie dojść do najgłębszych głębi swojej psychiki (znaleźć znajome ulice miasta) ale podświadomość mu na to nie pozwala (domy bez bram).
\subsection{Odpowiedz na pytania ziązane z motywem labiryntu w kulturze}
\subsubsection{Czym jest labitynt? Sformułuj jego definicję}
Labirynt - struktura budowlana mająca na celu zmylić i zgubić daną osobę poprzez zastosowanie wielu rozgałęzień, ślepych zaułków, etc.
\subsubsection{Jaką funkcję pełni labirynt w znanych Ci tekstach kultury}
W filmie Cube przedstawiony jest jako okazja na zgłębienie ludzkiej psychiki i poznanie, jak człowiek zachowuje się pod nadmierną presją i w oblicznu niewia\-domej w stosunku do innych ludzi.
\subsection{Wykonaj zadania dotyczące labiryntu w \emph{Sklepach cynamonowych}}
\subsubsection{Opisz labirynt}
Labirynt to znajome bohaterowi miasto, jednakże nieznajome mu uliczki z domami o dziwnej architekturze (brak bram, etc.)
\subsubsection{Podaj przyczyny obecności narratora w labiryncie}
Jest to eksploracja psychiki narratora, który chce poczuć wolność, oraz jednocześnie czuje się zagubiony i próbuje odnaleźć samego siebie.
\subsubsection{Nazwij odczucia bohatera spowodowane zaistniałą sytuacją}
Strach i odczucie ulgi po wyjściu z labiryntu (scena z dorożką)
\subsubsection{Określ, jaką funkcję pełni pora dnia, w której trakcie narrator odbywa wędrówkę}
Pora dnia nie jest stała, co dodaje do onirycznego przedstawienia otoczenia, w którym znajduje się narrator. Jest ona jednym z głównych powodów, dla których zaczynamy kwestionować czy to, co odczuwa bohater, to sen czy jawa.
\subsection{Przeczytaj podany tekst, a następnie uzupełnij tabelę na temat mityzacji rzeczywistości w opowiadaniu \\\emph{Sklepy Cynamonowe}}
\end{document}