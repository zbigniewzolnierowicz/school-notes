\documentclass[a4paper]{article}
\usepackage[T1]{fontenc}
\usepackage[polish,english]{babel}
\usepackage[utf8]{inputenc}
\usepackage{lscape}
\usepackage{graphicx}
\usepackage{booktabs}
\usepackage{float}
\usepackage{dirtytalk}
\begin{document}
\title{{\huge Lekcja 16} \\
{\large Nierzeczywista rzeczywistość świata opowiadań Bruno Schulza}}
\author{Zbigniew Żołnierowicz}
\date{13.11.2019}
\maketitle
\begin{enumerate}
    \item Ojcowskie brednie
    \item Wyjście do teatru
    \item Zapomniany portfel
    \item Opis sklepu cynamonowego
    \item Tylna ściana gimnazjum
    \item Wspomnienia lekcji rysunku
    \item Oddanie dorożki bohaterowi
    \item Podróż
    \item Odkrycie rany i porzucenie konia
    \item Spacer z kolegami ze szkoły i magia nocy
\end{enumerate}
\section{Karta pracy}
\subsection{Kim jest narrator opowiadania \emph{Sklepy cynamonowe}? Uzasadnij swoją odpowiedź.}
Narratorem opowiadania jest chłopiec w wieku gimnazjalnym lub ponadgimnazjalnum, co możemy wywnioskować z tego,
że pod koniec opowiadania spotyka się ze znajomymi z klasy i wspomina o zajęciach z rysunku w gimnazjum. Ma on ojca i matkę.
\\ Narrator jest subiektywny lub pamiętnikarski.
\pagebreak
\subsection{Przeczytaj przytoczony fragment opowiadania i prze\-analizuj zastosowane w nim środki stylistyczne.\\Wytłumacz, jaką pełnią funkcję w opisie i interpretacji świa\-ta.}
\begin{quotation}
    W okresie najkrótszych, sennych dni zimowych, ujętych z obu stron, od poranku i od wieczora, w futrzane krawędzie zmierzchów, gdy miasto rozgałęziało się coraz głębiej w labirynty zimowych nocy, z trudem przywoływane przez krótki świt do opamiętania, do\\powrotu[\dots]
\end{quotation}
% Please add the following required packages to your document preamble:
% \usepackage{graphicx}
\begin{table}[H]
    \resizebox{\textwidth}{!}{%
    \begin{tabular}{|l|l|l|}
    \hline
    \multicolumn{1}{|c|}{\textbf{Cytat}} & \multicolumn{1}{c|}{\textbf{Nazwa środka stylistycznego}} & \multicolumn{1}{c|}{\textbf{Funkcja}} \\ \hline
    \begin{tabular}[c]{@{}l@{}}``gdy miasto[\dots] z trudem przywoływane\\ do opamiętania''\end{tabular} & \begin{tabular}[c]{@{}l@{}}personifikacja,\\ onomatopeja\end{tabular} & Nadanie miastu osobowości \\ \hline
    \begin{tabular}[c]{@{}l@{}}``[\dots]labirynty zimowych nocy,[\dots]''\\ ``[\dots]futrzane krawędzie zmierzchów[\dots]''\end{tabular} & metafora & \begin{tabular}[c]{@{}l@{}}Wprowadzenie poczucia tajemnicy i\\ charakteryzacja nocy\end{tabular} \\ \hline
    ``senne dni zimowe'' & animizacja & przywołanie rzeczy do życia \\ \hline
    ``[\dots]futrzane krawędzie zmierzchów[\dots]'' & synestezja & działanie na zmysły \\ \hline
    \end{tabular}%
    }
\end{table}
\section{Wyjaśnij pojęcie \emph{oniryzm} i określ jego związek ze \emph{Sklepami cynamonowymi}}
Oniryzm to konwencja literacka, ukazująca rzeczywistość z pomieszaniem snu i jawy. W \emph{Sklepach cynamonowych} jest to ukazane m.in.: w postaci tytułowych sklepach cynamonowych, w których sprzedawane są niesamowite, egzotyczne i nierealne przedmioty, jak homunkulusy.
\end{document}