\documentclass[a4paper]{article}
\usepackage[T1]{fontenc}
\usepackage[polish,english]{babel}
\usepackage[utf8]{inputenc}
\usepackage{lscape}
\usepackage{graphicx}
\usepackage{booktabs}
\usepackage{float}
\usepackage{dirtytalk}
\begin{document}
\title{{\huge Lekcja 23} \\
{\large W obronie unicestwionych wartośći - ``Campo di Fiori'' Czesława Miłosza}}
\author{Zbigniew Żołnierowicz}
\date{14.02.2020}
\maketitle
\section{``Campo di Fiori'' - wiersz z tomu: ``Ocalenie'' (1945)}
\begin{itemize}
    \item Wyzbywa się tonacji martyrologicznej:
    \begin{quotation}
        (\dots)Zostawcie \\
        Poetom chwilę radości,\\
        Bo zginie nasz świat\\
        (``W Warszawie'')
    \end{quotation}
    \item Poszukuje porządku w chaosie świata:
    \begin{quotation}
        Czym jest poezja, która nie ocala\\
        Narodów ani ludzi?\\
        (``Przedmowa'')
    \end{quotation}
\end{itemize}
\subsection{Wyjaśnij znaczenie tytułu wiersza. Przypomnij po\-stać Giordana Bruno.}
\subsection{Podziel wiersz na trzy części i nadaj im tytuły}
\subsection{Wyjaśnij obrazy i zasady ich zestawienia w wierszu Miłosza:}
\subsubsection{plac rzymski Campo di Fiori (męczeński stos Giordana Bruno - 1600r.)}
\subsubsection{warszawszka karuzela; pożar Getta (1943r.)}
\section{Słowa-klucze, które można wykorzystać do interpretowania wiersza}
\begin{itemize}
    \item kontrast
    \item obrazy paralelne
    \item dynamika (czas gramatyczny, barwy)
    \item antynomie dźwięków i obrazów
    \item symbole
    \item topos
    \item dostojny rytm wiersza
\end{itemize}
\section{Jaką rolę wyzacza Czesław Miłosz poecie i po\-ezji? Zapisz wnioski w formie nakazów, uzupełniając rozpoczęte zdania:}
\begin{itemize}
    \item Nie godzić się na\dots/z tym\dots
    \item Dbać, aby\dots
    \item Wzniecać\dots
    \item Nie być\dots
\end{itemize}
\end{document}
