\documentclass[a4paper]{article}
\usepackage[T1]{fontenc}
\usepackage[polish,english]{babel}
\usepackage[utf8]{inputenc}
\usepackage{lscape}
\usepackage{graphicx}
\usepackage{booktabs}
\usepackage{float}
\usepackage{dirtytalk}
\begin{document}
\title{{\huge Lekcja 12} \\
{\large ``Ferdydurke'' jako powieść groteskowa}}
\author{Zbigniew Żołnierowicz}
\date{23.10.2019}
\maketitle
\section{Uzupełnij tabelę własnym komentarzem i przy\-kładami z tekstu}
\begin{table}[H]
    \resizebox{\textwidth}{!}{%
    \begin{tabular}{|l|l|}
    \hline
    \multicolumn{1}{|c|}{\textbf{Cecha groteski}} & \multicolumn{1}{c|}{\textbf{Realizacja w ``Ferdydurke''}} \\ \hline
    \begin{tabular}[c]{@{}l@{}}karykaturalne przejaskrawienie\\ rzeczywistości\end{tabular} & ``upupianie'', obraz nauczyciela łaciny \\ \hline
    \begin{tabular}[c]{@{}l@{}}zdeformowana rzeczywistość,\\ niezgodna z doświadczeniem\end{tabular} & ``gwałt przez uszy'' \\ \hline
    sprzeczność z ustalonymi normami & j.w. \\ \hline
    mieszanie tragizmu i komizmu &  \\ \hline
    zmienność nastrojów & \begin{tabular}[c]{@{}l@{}}nauczyciel widzi bawiące się dzieci,\\ które tak na prawde wypisują\\ wulgaryzmy na ścianach\end{tabular} \\ \hline
    niejednorodność stylowa & \begin{tabular}[c]{@{}l@{}}mieszanina prozy i dramatu\\ na lekcji łaciny\end{tabular} \\ \hline
    \begin{tabular}[c]{@{}l@{}}nagromadzenie sprzeczności i\\ absurdów\end{tabular} & synteza i analiza \\ \hline
    \end{tabular}%
    }
    \end{table}
\end{document}