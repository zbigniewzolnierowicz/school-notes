\documentclass[a4paper]{article}
\usepackage[T1]{fontenc}
\usepackage[polish,english]{babel}
\usepackage[utf8]{inputenc}
\usepackage{lscape}
\usepackage{graphicx}
\usepackage{booktabs}
\usepackage{float}
\usepackage{dirtytalk}
\begin{document}
\title{{\huge Lekcja 14} \\
{\large Rzecz o Godności - Porównujemy Soplicowo i Bolimowo}}
\author{Zbigniew Żołnierowicz}
\date{30.10.2019}
\maketitle
\begin{table}[H]
    \resizebox{\textwidth}{!}{%
    \begin{tabular}{|l|l|l|}
    \hline
     & \textbf{``Pan Tadeusz''} & \textbf{``Ferdydurke''} \\ \hline
    \begin{tabular}[c]{@{}l@{}}usytuowanie fragmentu\\ w całości utworu\end{tabular} & \begin{tabular}[c]{@{}l@{}}Obiad w zamku Soplicowo\\ po powrocie Tadeusza do domu, początek utworu\end{tabular} & \begin{tabular}[c]{@{}l@{}}Kolacja na dworzu\\ w Wolimowie u Kurleckich, koniec utworu\end{tabular} \\ \hline
    \begin{tabular}[c]{@{}l@{}}charakteryzacja\\ bohaterów\end{tabular} & \begin{tabular}[c]{@{}l@{}}Goście są grzeczni przy stole,\\Podkomorzy jest autorytetem\end{tabular} & \begin{tabular}[c]{@{}l@{}}Wszyscy są sztuczni, udają miłość,\\ odgrywają rolę grzecznych\end{tabular} \\ \hline
    motyw grzeczności & \begin{tabular}[c]{@{}l@{}}Zderzenie nowoczesnego myślenia\\ `Podkomorzego ze staromodnym\\ poglądem o grzeczności Sędzi\end{tabular} & \begin{tabular}[c]{@{}l@{}}Grzeczność udawana, wszyscy\\ odgrywają im przypisaną rolę,\\ wszystko robi się ``bo tak powinno\\ się robić''\end{tabular} \\ \hline
    język & \begin{tabular}[c]{@{}l@{}}Sędzia: Staromodny, gniewny, zdenerwowany\\ Narrator: obiektywny\end{tabular} & \begin{tabular}[c]{@{}l@{}}Bohaterowie: fałszywe\\ ``przymilanie się''\\ Narrator: subiektywny, w postaci Józia\end{tabular} \\ \hline
    \end{tabular}%
    }
\end{table}
\end{document}