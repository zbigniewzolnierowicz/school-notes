\documentclass[a4paper]{article}
\usepackage[T1]{fontenc}
\usepackage[polish,english]{babel}
\usepackage[utf8]{inputenc}
\usepackage{lscape}
\usepackage{graphicx}
\usepackage{booktabs}
\usepackage{float}
\usepackage{dirtytalk}
\begin{document}
\title{{\huge Lekcja 11} \\
{\large Dwie lekcje łaciny: Gombrowicza i Herberta}}
\author{Zbigniew Żołnierowicz}
\date{18.10.2019}
\maketitle
\section{Różnice między tekstami}
Podczas gdy tekst Herberta ``Lekcja Łaciny'' jest taki sam w układzie przez całą długość,
``Ferdydurke'' Gombrowicza pod koniec przeistacza się w dramat.

W ``Ferdydurke'', nauczyciel przedstawiony jest jako psychicznie chory (``uroił sobie nowy problem'') człowiek bez autorytetu o niskim morale,
który nie umie uczyć, poza tym, co wyczytuje w książce.

W ``Lekcji Łaciny'' nauczyciel jest przedstawiony jako osoba z autorytetem, której uczniowie się boją, ale zarazem respektują.
ale respektują.
\end{document}