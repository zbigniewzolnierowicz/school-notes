\documentclass[a4paper]{article}
\usepackage[T1]{fontenc}
\usepackage[polish,english]{babel}
\usepackage[utf8]{inputenc}
\usepackage{lscape}
\usepackage{graphicx}
\usepackage{booktabs}
\usepackage{float}
\usepackage{dirtytalk}
\begin{document}
\title{{\huge Lekcja 22} \\
{\large W jaki sposób teksty kultury ``gadają ze sobą'' - Na przykładzie ``Innego Świata'' Herlinga, Hasiora, Herberta, Kaczmarskiegi i Bugajskiego}}
\author{Zbigniew Żołnierowicz}
\date{17.01.2020}
\maketitle
\section{Piosenka Jacka Kaczmarskiego, Przemysława Gintrowskiego i Zbigniewa Łapińskiego ``Przesłuchanie anioła''}
\subsection{Dlaczego mówi się o Kaczmarskim jako o bardzie?}
Początek piosenki jest spokojny, ciągły i monotonny, kiedy podmiot liryczny zaczyna podchodzić do tytułowego anioła i opisuje jego wygląd.
Gdy wiersz zaczyna opisywać narzędzia przesłuchania, piosenka i wymowa słów nabiera siły i emocji.
Ten kontrast ma za zadanie pokazać energię i brutalność tego przesłuchania, a zarazem ``obedrzeć'' anioła z jego świętości.
Może to być pewnego rodzaju komentarz o taktykach, jakich Urząd Bezpieczeństwa używał do przesłuchiwania wrogów partii, ponieważ Jacek Kaczmarski był często opisywany jako bard Solidarności, lub też jako brutalność strażników w łagrach, nawiązując do ``Innego Świata.''
Funkcją muzyki Kaczmarskiego jest nadanie nastroju wierszowi, co nie zawsze jest możliwe w czysto pisanym dziele. Co więcej, ten utwór wywołuje emocje współczucia temu aniołowi, ponieważ sami czujemy tę ``brutalność'' czynów.
\end{document}
