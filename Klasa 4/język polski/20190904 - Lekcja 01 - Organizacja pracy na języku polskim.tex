\documentclass[a4paper]{article}
\usepackage[T1]{fontenc}
\usepackage[polish,english]{babel}
\usepackage{graphicx}
\usepackage[utf8]{inputenc}
\begin{document}
\title{{\huge Lekcja 01} \\
{\large Organizacja pracy na języku polskim}}
\author{Zbigniew Żołnierowicz}
\date{04.09.2019}
\maketitle
\section{Zasady współpracy na języku polskim}
\begin{enumerate}
    \item W ciągu semestru można uzyskać oceny cząstkowe. Są one świadectwem systematycznego przygotowania się do zajęć.
    \item Formy sprawdzania wiedzy i umiejętności 
    \begin{enumerate}
        \item Praca klasowa - 6
        \item Sprawdzian cytania ze zrozumieniem/sprawdzian wiadomości - 5
        \item Kartkówka powtórzeniowa - 2
        \item Próbna matura: ustna - 4; pisemna - 5
        \item Projekt - 3
        \item Aktywność na lekcji - 1/2
        \item Ćwiczenia/zadania praktyczne - 2
        \item Zadania domowe (ćwiczenia/wypracowania domowe) - 2/3
        \item Odpowiedź ustna - 3
        \item Recytacja - 2
        \item Konkursy pozaszkolne: udział - 2; etap rejonowy (2. etap) - 3/4; etap wojewódzki (3. etap) - 5/6
    \end{enumerate}
    \item Istnieje możliwość poprawiania ocen ze sprawdzianu i pracy klasowej
    \item Terminy pisania sprawdzianów w przypadku nieobecności
    \begin{itemize}
        \item usprawiedliwionej (choroba, reprezentowanie szkoły) - wg. decyzji nauczyciela
        \item nieusprawiedliwionej - w terminie wyznaczonym przez nauczyciela
        \item Procentowe progi na ocenę dopuszczającą:
        \begin{itemize}
            \item W pierwszym terminie: 41\%;
            \item W drugim terminie: 51\%;
        \end{itemize}
    \end{itemize}
    \item Ocena semestralna i roczna ustalana jest na podstawie średniej ważonej wynikającej z ocen
    cząstkowych. To podstawowe kryterium, jednak na\-uczyciel może podwyższyć ocenę, doceniając wkład pracy, dobrą frekwencję na lekcjach.
    \item Dwa razy w semestrze można zgłosić nieprzygotowanie do lekcji (nie dotyczy pisemnych prac domowych, sprawdzianów, terminowego przeczytania lektury).
    Zgłoszenie nieprzygotowania nie zwalnia z obowiązku samodzielnego opraowania materiału.
    \item Należy oddać wszystkie pracy domowe. Nieprzyniesienie zadania wiąże się z obowiązkiem dostarczenia pracy na kolejną lekcję.
    \item Uczeń ma obowiązek prowadzenia zeszytu przedmiotowego, który jest systematycznie kontrolowany przez nauczyciela. Ocenie podlegają samodzielnie prowadzone notatki.
    \item Uczeń zobowiązany jest do przynoszenia na lekcje tekstu literackiego: podręcznik, ksero tekstu, lektura. Uczeń, który nie spełni tego obowiązku, będzie musiał wykonać dodatkowe zadanie wskazane przez nauczyciela. Ocenie podlegają samodzielnie prowadzone notatki.
    \item Kłopoty ze zrozumieniem materiału, problemy z pisaniem prac pisemnych są podstawą do umówienia się z nauczycielem na indywidualne konsultacje.
    \item W ciągu semestru przewiduje się jedno wspólne, obowiązkowe wyjście do teatru.
\end{enumerate}
Oczywiście każda ocena ma charakter indywidualny i tylko od Ciebie oraz Twoich starań zależy sukces.
\section{Lektury}
\begin{itemize}
    \item ``Ferdydurke'' Witolda Gombrowicza
    \item ``Sklepy cynamonowe'' Bruno Schulza
    \item ``U nas w Auschwitzu'' Tadeusza Borowskiego
    \item ``Inny świat. Rozdział "Ręka w ogniu".'' Gustawa Herlinga-Grudzińskiego
    \item ``Zdążyć przed Panem Bogiem'' Hanny Krall
    \item ``Tango'' Sławomira Mrożka
\end{itemize}
\section{Sprawdziany}
\begin{itemize}
    \item 11.09.2019 - Praca klasowa z ``Ziemi Obiecanej''
    \item 20.10.2019 - Sprawdzian z wiadomości o dwudziestoleciu międzywojennym, Ferdydurke i Sklepach cynamonowych.
    \item koniec listopada 2019 - próbna matura pisemna (ocena za czytanie ze zrozumieniem i pracę klasową)
\end{itemize}
\section{Teatr}
Obowiązkowe wyjście do teatru: 25.09.2019 teatr nowy 19:30, trwa 90 minut
\end{document}