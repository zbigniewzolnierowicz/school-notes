\documentclass[a4paper]{article}
\usepackage[T1]{fontenc}
\usepackage[polish,english]{babel}
\usepackage{graphicx}
\usepackage[utf8]{inputenc}
\begin{document}
\title{{\huge Lekcja 02} \\
{\large ``Ziemia Obiecana'' Reymonta i Wajdy w pigułce}}
\author{Zbigniew Żołnierowicz}
\date{04.09.2019}
\maketitle
\section{Elementy świata przedstawionego}
\begin{description}
    \item[czas akcji] lata 80. XIX wieku
    \item[miejsce wydarzeń] Łódź
    \item[główni bohaterowie] Trzej bohaterowie, trzy wyznania, trzy narody
    \begin{itemize}
        \item Karol Borowiecki (Polak)
        \item Moryc Welt (Żyd)
        \item Maks Baum (Niemiec)
    \end{itemize}
\end{description}
\section{Najważniejsze sceny utworu}
\subsection*{Scena zakładania fabryki}
\subsection*{Scena śmierci Kesslera}
\subsection*{Scena strzelania do robotników}
\subsection*{Przejazd Anki ze wsi do miasta}
\section{Zagadnienia do interpretacji}
\subsection*{antagonizmy narodowościowe}
\subsection*{zróżnicowanie społeczne: ``wysadzeni z siodła'', ``wolni najmici''}
\subsubsection*{Wysadzeni z siodła}
Osoba która kiedyś miała ziemię, ale sprzedała ją i została mieszczanem.
Przy\-kładem tego jest Borowiecki, który wywodził się ze szlachty, ale za ziemię,
którą ojciec mu zostawił kupił fabrykę.
\subsection*{naturalizm}
Przedstawienie szarości fabryk, normalnego życia robotnika, materiały zapla\-mione krwią.
\subsection*{fin de si\'ecle = ``koniec wieku''}
\subsection*{modernizm}
Manifest nowoczesności. Przykładem są fabryki i wyparcie manufaktur przez nie.
\subsection*{kapitalizm}
\subsection*{antyurbanizm}
Przedstawienie miasta w negatywnym świetle.
\subsection*{funkcja tytułu}
Funkcja ironiczna
\subsection*{język postaci: stylizacja środowiskowa}
\section{Wybrane cytaty}
\begin{itemize}
    
    \item\begin{quote}
        ``Zakładamy fabrykę. Ja nie mam nic, ty nie masz nic, on nie ma nic... To razem właśnie mamy tyle, w sam raz, żeby założyć wielką fabrykę.''
    \end{quote}
    \item\begin{quote}
        \emph{Karol:} ``Jest pan maszyną, a nie człowiekiem''
    \end{quote}
    \item\begin{quote}
        \emph{Karol:} ``Tyle materiału na nic.''
    \end{quote}
    \item\begin{quote}
        ``Uczciwi mają dobre nie bo, po co im dobry czas.''
    \end{quote}
    \item\begin{quote}
        \emph{Maks:} ``Mój ojciec nie ma nic do stracenia oprócz honory, a tym towarem Łódź nie handluje.''
    \end{quote}
    \item\begin{quote}
        \emph{Trawiński:} ``Dla mądrych to jest dobry czas, kiedy będzie dla uczciwych?''
    \end{quote}
\end{itemize}
    
\end{document}