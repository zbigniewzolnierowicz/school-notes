\documentclass[a4paper]{article}
\usepackage[T1]{fontenc}
\usepackage[polish,english]{babel}
\usepackage[utf8]{inputenc}
\usepackage{lscape}
\usepackage{graphicx}
\usepackage{booktabs}
\usepackage{float}
\begin{document}
\title{{\huge Lekcja 05} \\
{\large Futuryzm i futuryści}}
\author{Zbigniew Żołnierowicz}
\date{18.09.2019}
\maketitle
\section{Futuryzm}
\begin{itemize}
    \item Fascynacja \textbf{M}iastem
    \item Sztuka \textbf{M}asowa
    \item Fascynacja \textbf{M}aszyną
    \item Agresywny
    \item Rozwijający się w krajach totalitarystycznych
\end{itemize}
\subsection{Przedstawiciele}
\begin{itemize}
    \item Bruno Jasieński - udał się do Rosji, zafascynowany komunizmem (tajemniczo ``zmarł'', ze względu na bycie członkiem inteligencji)
    \item Guillaume Apollinaire (Wilhelm Apolinary Kostrowicki)
\end{itemize}
\section{Karta pracy}
\subsection{Wypisz z tekstu cztery różne przykłady słownictwa perswazyjnego}
\begin{itemize}
    \item obowiązani są
    \item ma prawo
    \item zrywamy
    \item pszekreślamy
\end{itemize}
\subsection{Na podstawie pisowni przyjętej w przytoczonym ma\-nifeście ustal dwie zmiany, jakie futuryści chcieli wprowa\-dzić w polskiej ortografii. Podaj po jednym przykładzie wyrazu z tekstu, by potwierdzić słuszność swego wnios\-ku.}
\begin{table}[H]
    \centering
    \resizebox{\textwidth}{!}{%
    \begin{tabular}{@{}|l|l|@{}}
    \toprule
    \textbf{Postulowana zmiana w pisowni} & \textbf{Przykład} \\ \midrule
    Usunięcie opisów i onomatopei & \begin{tabular}[c]{@{}l@{}}``Zrywamy raz na zawsze z wszelkim \\ opisywańem (malarstwo), ale z drugiej\\ strony i z wszelkim onomatopeizowańem,\\ naśladowaniem głosuw pszyrody itp.''\end{tabular} \\ \midrule
    Zamiast zdań, krótkie zbitki słów & ``Pszekreślamy zdańe jako antypoezyjny dźiwoląg'' \\ \bottomrule
    \end{tabular}%
    }
\end{table}
\subsection{Do podanych postulatów dopisz cytaty, które je ilustrują}
\begin{enumerate}
    \item nakaz oryginalności \begin{quote}
        Każdy artysta obowiązany jest stwożyć zupełńe nową ńebywałą dotąd sztukę, kturą ma prawo nazwać swoim imieńem.
    \end{quote}
    \item esencjonalność, bogactwo treści przy wykorzystaniu minimalnej liczby słów \begin{quote}
        Kompozycję doskonałą, t. j. ekonomiczną i żelazną — minimum materjału pszy maximum ośągńętej dynamiki — nazywamy kompozycją futurystyczną. W ten sposub wolno jedyńe odtąd komponować.
    \end{quote}
    \item prawo artysty do własnego sposobu rozumowania \begin{quote}
        Dźeło sztuki uważamy za żecz dokonaną, konkretną i fizyczną. Kształt jego uwarunkowany jest śćiśle wewnętszną potrzebą.
    \end{quote}
    \item wyższość autonomicznych związków wyrazowych nad zdaniem \begin{quote}
        Pszekreślamy zdańe jako antypoezyjny dźiwoląg. Zdańe jest kompozycją pszypadkową, spojoną jedyńe słabym klejem drobnomieszczańskiej logiki. Na jego miejsce — skondensowane, ostre i konsekwentne kompozycje słuw, ńe krępowanyh żadnymi prawidłami składni logiki czy gramatycznośći, jedyńe twardą wewnętszną końecznością, ktura po tońe A domaga się tonu C, po tońe C tonu F i t. d.
    \end{quote}
    \item związek sztuki ze sprawami współczesnymi, bieżącymi, dziejącymi się tu i teraz \begin{quote}
        Dźeło sztuki jest ekstraktem. Rozpuszczone w szklance dńa powszedńego powinno ją całą zabarwić na swuj kolor.
    \end{quote}
\end{enumerate}
\end{document}