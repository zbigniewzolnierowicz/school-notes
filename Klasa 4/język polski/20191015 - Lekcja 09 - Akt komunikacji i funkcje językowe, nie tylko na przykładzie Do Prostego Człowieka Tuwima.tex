\documentclass[a4paper]{article}
\usepackage[T1]{fontenc}
\usepackage[polish,english]{babel}
\usepackage[utf8]{inputenc}
\usepackage{lscape}
\usepackage{graphicx}
\usepackage{booktabs}
\usepackage{float}
\usepackage{dirtytalk}
\begin{document}
\title{{\huge Lekcja 09} \\
{\large Akt komunikacji i funkcje językowe, nie tylko na przykładzie ``Do Prostego Człowieka'' Tuwima}}
\author{Zbigniew Żołnierowicz}
\date{15.10.2019}
\maketitle
\section*{Akt komunikacji}
Wg. Romana Jacobsona:
\begin{quote}
    Nadawca przekazuje komunikat do odbiorcy. Żeby ten komunikat dotarł, musi być wspólny kanał i wspólny kod. Musi być kontakt i wspólny kontekst.
\end{quote}
\section{Wstępne rozpoznanie problemu}
\subsection{Przyporządkuj podanym terminom właściwe opisy}
    \begin{description}
        \item[Funkcja impresywna] Celem wypowiedzi jest dążenie nadawcy do kształtowania określonych postaw i zachowań odbiorcy za pomocą nakazów, zakazów, komend, poleceń, próśb.
        \item[Funkcja fatyczna] Celem wypowiedzi jest podtrzymanie kontaktu
        \item[Funkcja ekspresywna] Celem wypowiedzi jest wyrażanie emocji, uczuć, woli, sądów nadawcy bez specjalnego ukierunkowania na ich odbiorcę.
        \item[Funkcja metajęzykowa] Rozmowa o języku
        \item[Funkcja poznawcza] Przekazujemy treść, która powinna być wspólna
        \item[Funkcja poetycka] 
    \end{description}
\subsection{Wykorzystując znajomość schematu komunikacji języ\-kowej, uzupełnij podane zdanie:}
\begin{quote}
    Stosując funkcję impresywną nadawca chce w swoim komunikacje {\tt przekonać} odbiorcę do określonych zachowań, natomiast stosując funkcję ekspresywną przekazuje odbiorcy swoje {\tt zdanie}.
\end{quote}
\subsection{Sformułuj tezę (hipotezę) swojej wypowiedzi}
\section{Rozwinięcie - część I. Interpretacja problemu}
\subsection{Do kogo i w jakim celu zwraca się osoba mówiąca w wierszu?}
\subsection{Wypisz z tekstu wiersza Tuwima po jednym przykła\-dzie podanych środków językowych. Określ, jakie każ\-dy z nich niesie ze sobą znaczenia. O czym świadczy użycie w tekście takich właśnie środków językowych?}
\begin{table}[]
    \centering
    \resizebox{\textwidth}{!}{%
    \begin{tabular}{@{}|l|c|l|@{}}
    \toprule
    \multicolumn{1}{|c|}{\textbf{Środek językowy}} & \textbf{Przykład z tekstu} & \multicolumn{1}{c|}{\textbf{Znaczenie dla wymowy tekstu}} \\ \midrule
    \textbf{Wykrzyknienie} & \textit{\say{Rżnij karabinem w bruk ulicy!}} & \begin{tabular}[c]{@{}l@{}}Funkcja impresywna\\ Nakaz, rozkaz, nawoływanie\end{tabular} \\ \midrule
    \textbf{Kolokwializm} & \textit{\say{I byle drab, byle szczeniak}} & \begin{tabular}[c]{@{}l@{}}Funkcja ekspresywna\\ Dostosowanie tekstu\end{tabular} \\ \midrule
    \textbf{Zgrubienie} & \textit{\say{A stado dzikich bab - kwiatami{[}...{]}}} & \begin{tabular}[c]{@{}l@{}}Funkcja ekspresywna\\ Wyrażenie negatywnych uczuć\end{tabular} \\ \midrule
    \textbf{Brutalizm} & \textit{\say{Rżnij karabinem w bruk ulicy!}} & \begin{tabular}[c]{@{}l@{}}Funkcja ekspresywna\\ Brutalne wyrażenie\end{tabular} \\ \midrule
    \textbf{Neologizm} & \textit{\say{Kiedy rozścierwi się, rozchami}} & Funkcja ekspresywna \\ \midrule
    \textbf{\begin{tabular}[c]{@{}l@{}}Wyraz\\ dźwiękonaśladowczy\end{tabular}} & \textit{\say{Rżnij karabinem w bruk ulicy!}} & \begin{tabular}[c]{@{}l@{}}Funkcja ekspresywna\\ Funkcja impresywna\\ \\ Podkreślić wyraz wiersza\end{tabular} \\ \midrule
    \textbf{Zdrobnienie} & \textit{\say{Obrzucać zacznie \say{żołnierzyków}}} & Funkcja ekspresywna \\ \bottomrule
    \end{tabular}%
    }
\end{table}
\section{Rozwinięcie - część II. Wykorzystanie kontekstów}
\subsection*{Który z utworów literackich byłby przydatny dla zilustro\-wania rozważanego przez Ciebie problemu?}
\section{Podsumowanie i sformułowanie wniosków. Refleksja własna.}
\end{document}