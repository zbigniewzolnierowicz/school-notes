\documentclass[a4paper]{article}
\usepackage[T1]{fontenc}
\usepackage[polish,english]{babel}
\usepackage[utf8]{inputenc}
\usepackage{lscape}
\usepackage{graphicx}
\usepackage{booktabs}
\usepackage{float}
\usepackage{dirtytalk}
\begin{document}
\title{{\huge Lekcja 17} \\
{\large Losy pokolenia Kolumbów}}
\author{Zbigniew Żołnierowicz}
\date{29.11.2019}
\maketitle
\section{Literatura wojny i okupacji}
\subsection{Ramy czasowe}
1939 - 1945
\subsection{Nazwy pokolenia}
\begin{description}
    \item[Kolumbowie] ``odkrywają inny świat'' - mają świadomość znaczenia wojny, więc muszą żyć szybko, szybko odkrywają życie
    \item[Pokolenie apokalipsy spełnionej] z odwołaniem do znaczenia biblijnego - apokalipsa się wydarzyła
    \item[Pokolenie apokalibsy zwyczajnej] apokalipsa to codzienność
    \item[Pokolenie młodych/dwudziestoletnich poetów Warszawy] 
\end{description}
\end{document}