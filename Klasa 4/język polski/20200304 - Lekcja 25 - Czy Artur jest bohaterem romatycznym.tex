\documentclass[a4paper]{article}
\usepackage[T1]{fontenc}
\usepackage[polish,english]{babel}
\usepackage[utf8]{inputenc}
\usepackage{lscape}
\usepackage{graphicx}
\usepackage{booktabs}
\usepackage{float}
\usepackage{dirtytalk}
\begin{document}
\title{{\huge Lekcja 24} \\
{\large Czy Artur jest bohaterem romantycznym?}}
\author{Zbigniew Żołnierowicz}
\date{04.03.2020}
\maketitle
\section{Podobieństwa Artura do Konrada (Dziady cz. III)}
\begin{itemize}
    \item Obaj się buntują
    \item Obaj uważają siebie za zbawców
    \item Obaj są indywidualistami
    \item Obaj są uwięzieni
    \item Obaj są wykształceni
    \item Obaj byli przeobrażeni
    \item Obaj ponoszą klęskę
\end{itemize}
\section{Motywy literackie}
\begin{itemize}
    \item Motywy antyczne \begin{itemize}
        \item Dedal i Ikar
        \item Prometeusz
        \item Syzyf
    \end{itemize}
    \item Motywy biblijne \begin{itemize}
        \item apokalipsa
        \item raj
    \end{itemize}
    \item Bóg, rozmowy z Bogiem \begin{itemize}
        \item -
    \end{itemize}
    \item rodzina \begin{itemize}
        \item -
    \end{itemize}
    \item matka \begin{itemize}
        \item -
    \end{itemize}
    \item ojciec \begin{itemize}
        \item -
    \end{itemize}
    \item kobieta \begin{itemize}
        \item -
    \end{itemize}
    \item dziecko, dzieciństwo \begin{itemize}
        \item -
    \end{itemize}
    \item młodość \begin{itemize}
        \item -
    \end{itemize}
    \item starość \begin{itemize}
        \item -
    \end{itemize}
    \item król, władca \begin{itemize}
        \item -
    \end{itemize}
    \item rycerz \begin{itemize}
        \item -
    \end{itemize}
    \item inteligent, intelektualista \begin{itemize}
        \item -
    \end{itemize}
    \item mistrz i uczeń \begin{itemize}
        \item -
    \end{itemize}
\end{itemize}
\end{document}
