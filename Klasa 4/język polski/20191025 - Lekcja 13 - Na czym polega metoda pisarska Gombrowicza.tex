\documentclass[a4paper]{article}
\usepackage[T1]{fontenc}
\usepackage[polish,english]{babel}
\usepackage[utf8]{inputenc}
\usepackage{lscape}
\usepackage{graphicx}
\usepackage{booktabs}
\usepackage{float}
\usepackage{dirtytalk}
\begin{document}
\title{{\huge Lekcja 12} \\
{\large `Na czym polega metoda pisarska Gombrowicza?}}
\author{Zbigniew Żołnierowicz}
\date{25.10.2019}
\maketitle
\section{Napisz dwuzdaniową interpretację stwierdze\-nia znajdującego się w zakończeniu powieści: ``Nie ma ucieczki przed gębą jak tylko w inną gębę''}
Człowiek nie ma własnego obrazu. Człowiek może tylko być w jakiejś formie narzuconej przez ooby trzecie.
\section{Przeczytaj uważnie krótki\\fragment \emph{``Ferdydurke''}}
\subsection{Przytoczony fragment ma charakter autotematyczny. Odtwórz własnymi słowami dwie informacje o meto\-dzie twórczej Gombrowicza}
Powtórzenie i rezygnacja ze spójności.
\subsection{Rozpoznaj zaplanowanego przez autora odbiorcę cytowanego wyżej fragmentu powieści. Nazwij go i wypisz fragment, który potwierdzi słuszność twojego wy\-boru.}
\begin{description}
    \item[Odbiorca] Inni pisarze i autorzy
    \item[Cytat na potwierdzenie] ``Powiedzcie mi, jak sądzicie[\dots]''
\end{description}
\section{Ustal, z jakim powszechnym wśród pisarzy \\ przeświadczeniem na temat ich własnej twórczości wyraźnie polemizuje Gombrowicz w przytoczonym fragmencie \emph{``Ferdydurke''}}
\begin{enumerate}
    \item Nieumiejętność interpretacji
    \item Przerywanie czytania
    \item Ograniczenie formy przekazu słowa
\end{enumerate}
\section{Wykorzystaj znajomość całej powieści i wskaż trzy słowa-klucze, następnie podaj ich znaczenie oraz przywołaj fakt fabularny, który się odwołuje do sensu kluczowego słowa}
\section{Wskaż cechy wspólne twórczości Leśmiana,\\Gombrowicza i Schulza.}
\begin{itemize}
    \item Przenikanie się snu i jawy
    \item Odejście od realizmu
    \item Estetyka groteski
\end{itemize}
\section{Podaj przykładowe wątki z \emph{``Ferdydurke''},\\które potwierdzają obecność w powieści wąt\-ków typowych dla literatury:}
\begin{description}
    \item[sensacyjnej] Ucieczka 
    \item[romansowej] Zuta, Pimko; Józio, Zosia
    \item[przygodowej] Poszukiwanie parobka
\end{description}
\section{Przedstaw głównego bohatera \emph{``Ferdydurke''}}
\begin{description}
    \item[Imię i nazwisko] Józio Kowalski
    \item[Wiek] trzydziestolatek
    \item[Zawód] pisarz
    \item[Pochodzenie społeczne] Inteligencja 
\end{description}
\section{W dwóch zdaniach wyjaśnij, na czym polega absurdalność sytuacji tego bohatera.}
Trzydziestoletni mężczyzna trafia do szkoły i jest traktowany jak dziecko.
\section{Do podanych ról społecznych (``gąb'') dopisz imiona lub nazwiska czy pseudonimy bohate\-rów \emph{``Ferdydurke''}, którzy dają sobie te ``gęby''\\przyprawić i postępują zgodnie z narzuconymi schematami}
\begin{description}
    \item[belfer zniewalający] Pimko
    \item[belfer przestraszony] Bladaczka
    \item[prymus] Syfon
    \item[uczeń chuligan, buntownik] Miętus
    \item[wyemancypowana pensjonarka] Zuta
    \item[dworski uwodziciel] Józio
    \item[grzeczna panienka] Zosia
\end{description}
\end{document}