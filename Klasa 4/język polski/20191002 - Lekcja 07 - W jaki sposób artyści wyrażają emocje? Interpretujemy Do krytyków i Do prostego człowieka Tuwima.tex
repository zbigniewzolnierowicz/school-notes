\documentclass[a4paper]{article}
\usepackage[T1]{fontenc}
\usepackage[polish,english]{babel}
\usepackage[utf8]{inputenc}
\usepackage{lscape}
\usepackage{graphicx}
\usepackage{booktabs}
\usepackage{float}
\begin{document}
\title{{\huge Lekcja 06} \\
{\large W jaki sposób artyści wyrażają emocje? Interpretujemy ``Do krytyków'' i ``Do prostego człowieka'' Tuwima}}
\author{Zbigniew Żołnierowicz}
\date{02.10.2019}
\maketitle
\section{Artykuł w czasopiśmie Skamander}
\subsection{Na podstawie pierwszego akapitu wyjaśnij, dla\-czego poeci tego ugrupowania stonią od jednoznacznej wy\-powiedzi programowej?}
\subsection{Napisz, jaka postawa jest obca poetom Skamandra.}
\subsection{Określ, jaką rolę skamandryci chcą odegrać dla odbiorców swoich wierszy.}
\subsection{Porównaj poglądy romantyków, młodopolan i \\ skamandrytów na kwestię sztuki i artysty.}
\begin{table}[H]
    \centering
    \resizebox{\textwidth}{!}{%
    \begin{tabular}{|l|l|l|}
    \hline
    \multicolumn{1}{|c|}{\textbf{Twórcy}} & \multicolumn{1}{c|}{\textbf{Sztuka}} & \multicolumn{1}{c|}{\textbf{Artysta}} \\ \hline
    romantycy & \begin{tabular}[c]{@{}l@{}}patriotyczna, narodowo wyzwoleńcza,\\ na pokrzepienie serc\end{tabular} & \begin{tabular}[c]{@{}l@{}}naznaczony przez Boga, walczący z nim,\\ utożsamiający się z Prometeuszem\end{tabular} \\ \hline
    młodopolanie &  & \begin{tabular}[c]{@{}l@{}}poza dobrem i poza złem, gardzi mieszczanami,\\ artysta biedny ale też lepszy niż filistrzy\end{tabular} \\ \hline
    skamandryci &  &  \\ \hline
    \end{tabular}%
    }
\end{table}
\section{``Do Krytyków'' Juliana Tuwima}
\subsection{Uzupełnij tabelę}
\begin{table}[H]
    \centering
    \resizebox{\textwidth}{!}{%
    \begin{tabular}{|l|l|}
    \hline
    \textbf{Rodzaj liryki} & liryka bezpośrednio-inwokacyjna  \\ \hline
    \textbf{Podmiot liryczny} & \begin{tabular}[c]{@{}l@{}}mężczyzna w mlodym wieku z chęcią do życia,\\ lubi wiosnę, prawdopodobnie poeta\end{tabular} \\ \hline
    \textbf{Adresant liryczny} & krytycy \\ \hline
    \textbf{Sytuacja liryczna} & jazda tramwajem w maju pod wpływem alkoholu \\ \hline
    \end{tabular}%
    }
\end{table}
\pagebreak
\subsection{Określ, co łączy utwór z założeniami tworzenia poezji wg. skamandrytów}
\begin{itemize}
    \item tematyka codzienna
    \item aktywizm
    \item witalizm
    \item bergsonizm
    \item optymizm
    \item kolokwializmy
\end{itemize}
\subsection{Odpowiedz, dlaczego wiersz jest skierowany do krytyków}
Jest manifestem skamandryzmu i prowokacją ludziom przeciwnych temu kierun\-kowi
\subsection{Opisz, jaka konepcja poezji wyłania się z utworu \\ Tuwima}
Poezja ma być prosta, dedykowana dla prostego człowieka, dotycząca spraw powszednich.
\end{document}