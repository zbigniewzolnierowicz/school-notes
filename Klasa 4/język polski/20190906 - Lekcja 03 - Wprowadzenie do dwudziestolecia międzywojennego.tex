\documentclass[a4paper]{article}
\usepackage[T1]{fontenc}
\usepackage[polish,english]{babel}
\usepackage[utf8]{inputenc}
\usepackage{lscape}
\usepackage{graphicx}
\usepackage{booktabs}
\usepackage{float}
\begin{document}
\title{{\huge Lekcja 03} \\
{\large Wprowadzenie do dwudziestolecia międzywojennego}}
\author{Zbigniew Żołnierowicz}
\date{06.09.2019}
\maketitle
\section*{Fascynacja techniką}
\begin{itemize}
    \item Czołgi
    \item Karabiny maszynowe
    \item Radiotelegraf
    \item Gaz musztardowy
    \item Upowszechnienie kina
    \item Upowszechnienie samochodu
    \item Tramwaje elektryczne
\end{itemize}
\section*{Zmiany polityczne}
Dojście do władzy totalitarystycznych doktryn politycznych, przez biedę powojenną.
\section*{Filozofia}
\begin{itemize}
    \item Albert Einstein - jego ogólna teoria względności zainspirowała wielu autorów: 
    \begin{itemize}
        \item Marcel Proust
        \item Bruno Schulz
    \end{itemize}
    \item Sigmunt Freud - psychoanaliza. Wprowadził 3 pojęcia do analizy, co się dzieje z człowiekiem:
    \begin{itemize}
        \item ego - ``ja'': warstwa, na którą działają wszystkie czynniki zewnętrzne
        \item super ego - ``super ja'': tradycje, zwyczaje, savoir vivre, zasady\newline moralne; wszystko, co nas tłamsi i nas ogranicza.
        \item id - ``to'': wewnętrzne zwierzę, zawsze złe drugie oblicza człowieka, ujawniające się przez sny i przejęzyczenia lub hipnozę.
    \end{itemize}
\end{itemize}
\section*{Sztuka}
\begin{itemize}
    \item Kubizm: rysowanie kształtami \begin{itemize}
        \item Picasso
    \end{itemize}
    \item Surrealizm (nadrealizm): zapisywanie podświadomości \begin{itemize}
        \item Salvador Dal\'i
    \end{itemize}
    \item Futuryzm: fascynacja techniką przyszłości
    \item Awangarda: z francuskiego \emph{avant garde}, znaczącego straż przednia;
    oznacza artystów, którzy są przodownikami sztuki
    \item Dadaizm: fascynacja wolnością, jaką ma dziecko; zabawa sztuką\begin{itemize}
        \item Tristan Tzara
    \end{itemize}
\end{itemize}
\pagebreak
\section*{Karta pracy}
\subsection{Uzupełnij informacje o zmianach, jakie zaszły w świe\-cie w okresie dwudziestolecia międzywojennego}
\begin{table}[H]
    \centering
    \resizebox{\textwidth}{!}{
    \begin{tabular}{@{}|l|l|l|@{}}
    \toprule
    \textbf{Dziedzina}   & \textbf{Zmiana} & \textbf{Oddziaływanie}                      \\ \midrule
    \textit{polityka}    & totalitaryzm    & popularyzacja nazizmu i faszyzmu            \\ \midrule
    \textit{nauka}       & Einstein        & wprowadzenie ogólnej teorii względności     \\ \midrule
    \textit{psychologia} & freudyzm        & wprowadzenie psychoanalizy                  \\ \midrule
    \textit{kultura}     & kultura masowa  & pauperyzacja sztuki, ``sztuka dla ubogich'' \\ \bottomrule
    \end{tabular}}
\end{table}
\subsection{Podaj nazwiska twórców przypisanych do wymienio\-nych nurtów, przykładowe tytuły ich dzieł oraz informacje o sposobie przekazu i obrazie świata}
% Please add the following required packages to your document preamble:
% \usepackage{booktabs}
% \usepackage{lscape}
\begin{landscape}
    \begin{table}[H]
    \centering
        \begin{tabular}{@{}|l|l|l|l|l|@{}} 
        \toprule
            \emph{Nazwa nurtu} & \emph{Nazwiska twórców} & \emph{Tytuły dzieł} & \emph{Sposób przekazu} & \emph{Obraz świata} \\ \midrule
            \textbf{dadaizm} & Marcel Duchamp & \begin{tabular}[c]{@{}l@{}}``Fontanna'',\\ ``Mona Lisa z wąsem''\end{tabular} & ``ready-made'' & antyestetyzm \\ \midrule
            \textbf{kubizm} & Pablo Picasso, Władimir Majakowski & ``Panny z Awinionu'' & geometria & odrzucenie formy ludzkiej \\ \midrule
            \textbf{surrealizm} & \begin{tabular}[c]{@{}l@{}}Salvador Dali,\\ Luis Benuel,\\ Guillaume Apollinaire,\\ André Breton,\\ René Magritte\end{tabular} & \begin{tabular}[c]{@{}l@{}}``Pies andaluzyjski'',\\ ``Kaligramy'',\\ ``Płonąca żyrafa''\end{tabular} & język pozarozumowy & absurd \\ \midrule
            \textbf{futuryzm} & \begin{tabular}[c]{@{}l@{}}Filippo Tommaso Marinetti,\\ Władimir Majakowski,\\ Stanisław Młodożeniec,\\ Tytus Czyżewski,\\ Bruno Jasieński\end{tabular} & \begin{tabular}[c]{@{}l@{}}``But w butonierce'',\\ ``Mechaniczny ogród'',\end{tabular} & ruch & przeczucie katastrofy \\ \midrule
            \textbf{ekspresjonizm} & F. W. Murnau & ``Nosferatu - Symfonia Grozy'' &  & chwilowość \\ \bottomrule
        \end{tabular}
    \end{table}
\end{landscape}
\end{document}