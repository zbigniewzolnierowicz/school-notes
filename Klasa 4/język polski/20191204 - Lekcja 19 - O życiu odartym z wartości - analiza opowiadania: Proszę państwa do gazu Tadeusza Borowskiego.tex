\documentclass[a4paper]{article}
\usepackage[T1]{fontenc}
\usepackage[polish,english]{babel}
\usepackage[utf8]{inputenc}
\usepackage{lscape}
\usepackage{graphicx}
\usepackage{booktabs}
\usepackage{float}
\usepackage{dirtytalk}
\begin{document}
\title{{\huge Lekcja 19} \\
{\large O życiu odartym z wartości - analiza opowiadania: ``Proszę państwa do gazu'' Tadeusza Borowskiego}}
\author{Zbigniew Żołnierowicz}
\date{04.12.2019}
\maketitle
\section{Interpretacja tytułu}
Tytuł opowiadania ma charakter ironiczny. Słowa te mógłby wypowiadać esesman, który zaprasza do śmierci (oksymoron - zapraszanie na śmierć) lub jeden z więźniów, człowiek \textbf{zlagrowany} (ten, który przyjął już prawa obozu i uważa, że są na porządku dziennym).
\section{Prawa obozu}
Charakter praw jest bardzo prymitywny i brutalny. Nie śledzenie tych niespi\-sanych praw jest równe ze śmiercią.
\begin{enumerate}
    \item Gdy osoba o wysokim statusie trafia do obozu, jest równana z osobami o możliwie jak najniższym statusie
    \item Istnieje handel wewnątrzobozowy
    \item Jedzenie to praktycznie waluta
    \item Ci, którzy idą na śmierć są oszukiwani aż do przysłowiowego stryczka
\end{enumerate}
\section{Rady}
\begin{itemize}
    \item Nie bierz ze sobą pieniędzy ani ubrań, tylko koszulę jedwabną z kołnierzem, z koszulą gimnastyczną pod spodem
    \item Jeżeli masz jakieś kosztowności, zostaniesz rozstrzelany
\end{itemize}
\end{document}
