\documentclass[a4paper]{article}
\usepackage[T1]{fontenc}
\usepackage[spanish,english,polish]{babel}
\usepackage{graphicx}
\usepackage{booktabs}
\usepackage[normalem]{ulem}
\usepackage{float}
\usepackage{indentfirst}
\useunder{\uline}{\ul}{}
\usepackage[utf8]{inputenc}
\usepackage{minted}
\begin{document}
\title{{\huge Lekcja 04} \\
    {\Large Typy zmiennych\linebreak Wartości typów podstawowych (string, number),\linebreak ich metody i atrybuty}}
\author{Zbigniew Żołnierowicz}
\date{30.09.2019}
\maketitle
Wartości typów podstawowych takich jak string czy number nie posiadają swoich metod i atrybutów, ponieważ nie są obiektami. JavaScript pozwala jednak wywoływać metody i atrybuty na tych wartościach, wtedy traktowane są jak obiekty.
\begin{minted}{javascript}
let text = "Ala ma Kota";
text.length // 11
text.indexOf('a') // 2
text.search('ma') // 4
text.lastIndexOf('a') // 10
text.substr(2, 3) // a m
text.slice(2, 8) // a ma K
text.replace('A','U') // Alu ma Kota
text.toUpperCase() // ALA MA KOTA
text.toLowerCase() // ala ma kota
text.concat(' a Jasio ma psa') // Ala ma Kota a Jasio ma psa
text.charAt(0) // A
text.charCodeAt(0) // 65
text.split(' ') // [ 'Ala', 'ma', 'Kota' ]
\end{minted}
\end{document}