\documentclass[a4paper]{article}
\usepackage[T1]{fontenc}
\usepackage[spanish,english,polish]{babel}
\usepackage{graphicx}
\usepackage{booktabs}
\usepackage[normalem]{ulem}
\usepackage{float}
\usepackage{indentfirst}
\useunder{\uline}{\ul}{}
\usepackage[utf8]{inputenc}
\usepackage{minted}
\usemintedstyle{xcode}
\begin{document}
\title{{\huge Lekcja 08} \\
    {\Large Tablice}}
\author{Zbigniew Żołnierowicz}
\date{18.11.2019}
\maketitle
\section{Metody tablic}
\begin{minted}[breaklines, tabsize=2]{javascript}
    let arr = []
    arr.push(x) // Dodaje element na koniec tablicy
    arr.pop() // Zwraca element z końca tablicy i usuwa go z tablicy
    arr.shift() // Zwraca element z początku tablicy i usuwa go
    arr.unshift(x) // Dodaje element na początek tablicy
    let accumulated = arr.reduce((element, accumulator) => {
        accumulator += element 
    }) // Skupia wszystkie wartości w jednej zmiennej bazując na funkcji
    arr.forEach((element, index) => {
        console.log(`${element} on index ${index}`)
    }) // Iteruje na tablicy
    arr.map(element => element += 1) // Zmienia wartość elementów w tablicy używając podanej funkcji
    arr.sort((a, b) => { // Sortuje tablicę
        if (a > b) return 1; // Jeżeli pierwsza wartość jest większa niż druga, zwróc liczbę dodatnią
        else if (a < b) return -1; // Jeżeli druga wartość jest większa niż pierwsza, zwróć liczbę ujemną
        else return 0; // jeżeli są równe, zwróć 0
    })
    arr.join('-') // Zwraca ciąg znaków z elementów tablicy połączony separatorem podanym jako parametr
    arr.concat(arr2) // Łączy dwie tablice ze sobą
    arr.find(x => x % 2 === 0) // Zwraca pierwszą wartość dla której funkcja podana jako argument zwraca true
    arr.reverse() // Odwraca tablicę
    arr.every(x => x % 2 === 0) // Sprawdza, czy funkcja zwraca true dla każdej wartości, jeśłi tak - zwraca true, w przeciwnym razie - zwraca false
    arr.some(x => x % 2 === 0) // j.w. ale sprawdza czy przynajmniej jeden element spełnia tę wartość
    arr.flat(x) // Jeżeli tablica ma inne tablice jako element, zwraca tablicę odniżoną o x wymiarów
    arr.filter() // Zwraca tablicę wyłącznie z elementami, dla których funkcja podana jako argument zwraca true
\end{minted}
\section{Sortowanie tablicy}
\begin{minted}{javascript}
    let tab = [15982519, 0, -1, 949494949, 2]
    tab.sort((a, b) => -1 * (a - b))
    console.log(tab) // [ -1, 0, 2, 15982519, 949494949 ]
\end{minted}
\section{Łączenie tablic}
\begin{minted}{javascript}
    let tab1 = [15982519, 0, -1, 949494949, 2]
    let tab2 = [44444444, 101001010]
    let tab = tab1.concat(tab2)
    console.log(tab) // [ 15982519, 0, -1, 949494949, 2, 44444444, 101001010 ]
\end{minted}
\section{Przedstawienie tablicy jako ciąg znaków}
\begin{minted}{javascript}
    let tab1 = [15982519, 0, -1, 949494949, 2]
    console.log(tab1.join(', ')) // 15982519, 0, -1, 949494949, 2
\end{minted}
\end{document}