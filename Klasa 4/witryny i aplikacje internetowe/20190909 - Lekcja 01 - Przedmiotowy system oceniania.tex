\documentclass[a4paper]{article}
\usepackage[T1]{fontenc}
\usepackage[spanish,english,polish]{babel}
\usepackage{graphicx}
\usepackage{booktabs}
\usepackage[normalem]{ulem}
\usepackage{float}
\useunder{\uline}{\ul}{}
\usepackage[utf8]{inputenc}
\begin{document}
\title{{\huge Lekcja 01} \\
    {\large Przedmiotowy system oceniania}}
\author{Zbigniew Żołnierowicz}
\date{09.09.2019}
\maketitle
Każdy musi zdobyć minimalną liczbę ocen (po 4 na semestr). W razie oceny niedostatecznej ze sprawdzianu uprasza się o 1 poprawę.
\section*{Wykład semestr 1}
\begin{itemize}
    \item 2 sprawdziany \begin{itemize}
        \item pierwszy z wiedzy z klasy 3 (23 września): waga 4, waga poprawy 4
        \item drugi z JS: waga 4, waga poprawy 4
    \end{itemize}
    \item Min. 1 kartkówka wagi 2 (trzy ostatnie tematy)
    \item Min. 1 ocena z aktywności/ćwiczenia: waga 1/2
    \item Możliwa ocena za udział i osiągnięcia w konkursach/olimpiadach: waga 1-4
    \item 1 nieprzygotowanie (zgładszane po sprawdzeniu obecności)
\end{itemize}
\section*{Wykład semestr 2}
\begin{itemize}
    \item 2 sprawdziany \begin{itemize}
        \item JS: waga 4, waga poprawy 4
        \item Egzamin próbny pisemny (w tygodniu 24-28 lutego 2020) - do egzaminu trzeba podejść, nie trzeba zaliczyć: waga 4
    \end{itemize}
    \item Min. 1 kartkówka wagi 2 (trzy ostatnie tematy)
    \item Min. 1 ocena z aktywności/ćwiczenia: waga 1/2
    \item 1 nieprzygotowanie (zgładszane po sprawdzeniu obecności)
    \item Możliwa ocena za udział i osiągnięcia w konkursach/olimpiadach: waga 1-4
\end{itemize}
\section*{Uwagi}
Szczęśliwy numerek zwalnia z odpowiedzi, ale nie z zapowiedzianej kartkówki, sprawdzianu i aktywności na zajęciach.

Notatki w formie tradycyjnej lub elektronicznej są obowiązkowe (osoba wskazana i poproszona o pokazanie notatek - w przypadku ich braku, lub niekompletności w nich otrzymuje ocenę niedostateczną za aktywność)

W przypadku nieobecności na sprawdzianie lub zapowiedzianej kartkówce i nieusprawiedliwieniu swojej nieobecności u nauczyciela - sprawdzian będzie oceniany w wyższych progach procentowych

Ocena roczna jest wystawiana na podstawie średniej ważonej wszystkich ocen z obu części wykładu (JS i PHP)
\end{document}