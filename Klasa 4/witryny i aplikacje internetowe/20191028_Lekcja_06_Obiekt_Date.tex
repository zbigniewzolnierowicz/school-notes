\documentclass[a4paper]{article}
\usepackage[T1]{fontenc}
\usepackage[spanish,english,polish]{babel}
\usepackage{graphicx}
\usepackage{booktabs}
\usepackage[normalem]{ulem}
\usepackage{float}
\usepackage{indentfirst}
\useunder{\uline}{\ul}{}
\usepackage[utf8]{inputenc}
\usepackage{minted}
\begin{document}
\title{{\huge Lekcja 06} \\
    {\Large Obiekt Date}}
\author{Zbigniew Żołnierowicz}
\date{28.10.2019}
\maketitle
Obiekt \mintinline{javascript}{Date} posiada wiele metod do ustawiania, pobierania i manipulacji datami.
Nie posiada żadnych właściwości. Należy pamiętać, że czas i data są pobierane z urządzenia końcowego (komputer, smartfon, tablet), a nie z serwera.
Aby korzystać z \mintinline{javascript}{Date()} w kodzie skryptu należy utworzyć najpierw nowy obiekt.
\begin{minted}{javascript}
    let data = new Date();
    alert(data);
\end{minted}
\section{Metody}
Metody obiektu \mintinline{javascript}{Date} są do obsługi daty i czasu zawartego w tych obszernych kategoriach:
\begin{description}
    \item[set] służy do ustawienia wartości daty i czasu w obiektach \mintinline{javascript}{Date}
    \item[get] służy do pobierania wartości daty i czasu w obiektach \mintinline{javascript}{Date}
    \item[to] służy do powtórzenia łańcuchów wartości z obiektów \mintinline{javascript}{Date}
\end{description}
Parsowanie i metody UTC służą do analizy składniowej łańcuchów \mintinline{javascript}{Date}.
\end{document}