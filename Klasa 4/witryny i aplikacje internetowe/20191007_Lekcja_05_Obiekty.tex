\documentclass[a4paper]{article}
\usepackage[T1]{fontenc}
\usepackage[spanish,english,polish]{babel}
\usepackage{graphicx}
\usepackage{booktabs}
\usepackage[normalem]{ulem}
\usepackage{float}
\usepackage{indentfirst}
\useunder{\uline}{\ul}{}
\usepackage[utf8]{inputenc}
\usepackage{minted}
\begin{document}
\title{{\huge Lekcja 05} \\
    {\Large Obiekty}}
\author{Zbigniew Żołnierowicz}
\date{07.10.2019}
\maketitle
\section{Obiekty}
W języku JavaScript wszystko poza prymitywnymi wartościami ({\tt string}, {\tt boolean}, {\tt number}, {\tt null} i {\tt undefined}) jest obiektem. Obiekt to struktura danych złożona z zera, jednej lub wielu nazwanych zmiennych, przechowujących wartości (atrybuty i właściwości) i funkcje (metody).

\subsection*{JSON}
W języku JavaScript do definiowania obiektów wykorzystuje się notację JSON (\emph{JavaScript Object Notation}). Obiekty nie są strukturami stałymi. Można je rozszerzać o atrybuty i metody w dowolnym momencie.
\begin{minted}{javascript}
    let car = {
        color: 'Red',
        brand: 'Mazda',
        productionYear: 1991
    };
    console.log(typeof car);
    // 'object'
    console.log(car.productionYear);
    // 1991
    console.log(car.color);
    // 'Red'
    console.log(car.brand);
    // 'Mazda'
\end{minted}
\pagebreak
\subsection*{Stwórz metodę 'Powitanie'. Poprzez tą metodę ma pojawić się 'Witaj, masz to auto: '}
\begin{minted}[tabsize=2,breaklines]{javascript}
    let car = {
        color: 'Red',
        brand: 'Mazda',
        productionYear: 1991,
        hello() {
            console.log(`Hello. You have a ${this.color} ${this.brand} made in ${this.productionYear}`)
        }
    };
    car.hello();
    // Hello. You have a Red Mazda made in 1991
\end{minted}
W języku JavaScript słowo kluczowe {\tt this} jest referencją do obiektu macierzystego względem metody, w której jest wywoływane.

Język JavaScript umożliwia tworzenie prototypów (konstruktorów) obiektów w postaci funkcji generujących obiekty.
\begin{minted}[tabsize=2,breaklines]{javascript}
    class Uczen {
        constructor(imie, oceny, znajomi) {
            this.imie = imie;
            this.oceny = oceny;
            this.znajomi = znajomi;
        }
        srednia = () => this.oceny.reduce((a, b) => a + b) / this.oceny.length;
    }
\end{minted}
\begin{description}
    \item[\mintinline{javascript}{Math.LN10}] stała wartość; logarytm naturalny z 10
    \item[\mintinline{javascript}{Math.LN2}] stała wartość; logarytm naturalny z 2
    \item[\mintinline{javascript}{Math.PI}] stała wartość; liczba $\pi$
    \item[\mintinline{javascript}{Math.SQRT1_2}] stała wartość; pierwiastek kwadratowy z $\frac{1}{2}$
    \item[\mintinline{javascript}{Math.SQRT2}] stała wartość; pierwiastek kwadratowy z 2   
\end{description}
\end{document}