\documentclass[a4paper]{article}
\usepackage[T1]{fontenc}
\usepackage[spanish,english,polish]{babel}
\usepackage{graphicx}
\usepackage{booktabs}
\usepackage[normalem]{ulem}
\usepackage{float}
\usepackage{indentfirst}
\useunder{\uline}{\ul}{}
\usepackage[utf8]{inputenc}
\begin{document}
\title{{\huge Lekcja 02} \\
    {\large Jak komputer wykonuje program?}}
\author{Zbigniew Żołnierowicz}
\date{09.09.2019}
\maketitle
\textbf{Program} komputerowy to ciąg symboli tworzących instrukcje, opisujących realizowane obliczenia zgodnie z przyjętymi regułami, zwanymi językiem programowania.

\textbf{Aplikacja} to program lub zbiór programów realizujących określone działania (np. Microsoft Office Word)
\section{Języki kompilowalne i interpretowalne}
\subsection{Języki wysokiego poziomu}
\begin{itemize}
    \item Abstrakcyjne (np. języki obiektowe)
    \item Niezależne od architektury
    \item Łatwe do zrozumienia przez człowieka
\end{itemize}
\subsection{Języki niskiego poziomu}
\begin{itemize}
    \item Ściśle związane z bieżącą architekturą
    \item Trudne do zrozumienia przez człowieka
\end{itemize}
\subsection{Translator}
\textbf{Translator} to program komputerowy, który przekształca kod w pewnym określonym języku programowania na kod w innym języku programowania (zwykle tego samego lub wyższego poziomu). Proces takiego przekształcenia nazywamy \textbf{translacją}.
\pagebreak
\subsection{Języki kompilowalne}
\textbf{Kompilator} to program komputerowy, który przekształca kod w pewnym określonym języku programowania wyższego poziomu na kod w języku programowania niższego poziomu. Proces takiego przekształcenia nazywamy \textbf{kompilacją}. W szczególności, kompilator może przekształcać kod w danym języku programowania na kod w języku maszynowym.

Każdy język programowania, dla którego istnieje kompilator nazywamy językiem kompilowalnym.
\subsubsection{Przykłady języków kompilowalnych}
\begin{itemize}
    \item C
    \item C++
    \item C\#
    \item Java
    \item Język assemblera
\end{itemize}
Uwaga: każdy kompilator jest translatorem, ale nie kazdy translator jest kompilatorem.
\subsection{Assembler}
\textbf{Assembler} to kompilator, który przekształca kod w języku assemblera na kod w języku maszynowym. Proces takiego przekształcenia nazywamy \textbf{assemblacją}.
\subsection{Kompilacja}
\textbf{Kompilacją} nazywamy również ciąg procesów przekształcających kod źródłowy w plik wykonywalny, gotowy do uruchomienia na maszynie. Standardowy proces kompilacji składa się z czterech kroków.
\begin{enumerate}
    \item[\textbf{Krok 1}]\emph{Prekompilacja}: Pliki z kodem źródłowym są przygotowywane do etapu kompilacji
    \item[\textbf{Krok 2}]\emph{Kompilacja}: Pliki z kodem źródłowym są kompilowane do języka assemblera.
    \item[\textbf{Krok 3}]\emph{Assemblacja}: Pliki z kodem źródłowym w języku assemblera przetwarzane są na język maszynowy.
    \item[\textbf{Krok 4}]\emph{Linkowanie}: Program zwany linkerem łączy wszystkie pliki w jeden plik wykonywalny.
\end{enumerate}
\pagebreak
\subsection{Języki interpretowalne}
\textbf{Interpreter} to program komputerowy, który analizuje, interpretuje i natychmiast wykonuje kod w pewnym określonym języku programowania. Proces ten nazywamy interpretacją.

Każdy język programowania, dla którego istnieje interpreter nazywamy \textbf{językiem interpretowalnym} (skryptowym). Źródło zawierające kod w języku interpretowalnym nazywamy \emph{skryptem}.

\subsubsection{Przykłady języków interpretowalnych}
\begin{itemize}
    \item Python
    \item JavaScript
    \item Ruby
    \item Shell script
    \item PHP
\end{itemize}
\subsection{Debugger}
\textbf{Debugger} to program komputerowy służący do dynamicznej analizy działania innych programów. Analiza ta, obejmująca obserwację wykonywanych instrukcji i zawartości zmiennych, pozwala zidentyfikować błędy występujące w badanym programie.
\subsection{Aplet}
\textbf{Aplet} to program komputerowy osadzony na stronie internetowej, uruchamany (np. aplet uruchamiany w wirtualnym środowisku Javy) lub interpretowany (np. kod w języku JavaScript) przez przeglądarkę internetową.
\section{Zadanie domowe}
Czym jest ``cykl życia aplikacji''?
\end{document}