\documentclass[a4paper]{article}
\usepackage[T1]{fontenc}
\usepackage[spanish,english,polish]{babel}
\usepackage{graphicx}
\usepackage{booktabs}
\usepackage[normalem]{ulem}
\usepackage{float}
\usepackage{indentfirst}
\useunder{\uline}{\ul}{}
\usepackage[utf8]{inputenc}
\begin{document}
\title{{\huge Lekcja 03} \\
    {\large }}
\author{Zbigniew Żołnierowicz}
\date{16.09.2019}
\maketitle
\section{Analiza i projektowanie}
Zamawiający oprogramowanie i wykonujący je uzgadniają szczegóły dotyczące wspólnych celów. Ustalone wtedy zostają wykorzystane technologie. Powstaje specyfikacja funkcjonalna, czyli dokładny opis funkcjonalności, jakie ma udostępnić aplikacja i procedur, którym ma podlegać.
\section{Implementacja}
Implementacja to realizacja funkcjonalności w procesie programowania, budowa wykorzystywanych baz danych i debuggowanie powstałego kodu.
\section{Testowanie i weryfikacja}
\begin{itemize}
    \item Testowanie może się odbywać fragmentami
    \item Sprawdzenie zgodności z dokumentacją projektową (weryfikacja) i prawidłowości działania
    \item Sprawdzenie odporności nabłędy użytkownika
    \item Blisko warunków produkcyjnych 
\end{itemize}
\section{Dokumentacja}
\subsection{Dokumentacja projektu}
Zawiera opis projektu, wymogi, np. kontrakt, specyfikacja funkcjonalna i/lub harmonogram płac.
\subsection{Dokumentacja programistyczna}
Opisuje ona zasady działania utworzonego kodu.
\subsection{Dokumentacja użytkownika}
Opisuje zasady użytkowania aplikacji.
\section{Wdrożenie}
\begin{itemize}
    \item Instalacja aplikacji w środowisku produkcyjnym (uruchomieniowym)
    \item Migracja danych ze starych aplikacji
    \item Szkolenie użytkowników
\end{itemize}
\section{Wsparcie i utrzymanie}
\begin{itemize}
    \item Naprawa błędów niewykrytych w czasie testów
    \item Wsparcie użytkowników
    \item Aktualizacja oprogramowania (np. dostosowanie do zmian w prawie)
\end{itemize}
\section{Zadanie}
\begin{itemize}
    \item aplikacja na telefon
    \item uczeń wybiera przedmiot
    \item wybiera rozdział
    \item wpisuje krótki opis
    \item dopóki nie odznaczy że zrobi, to jest permanentne powiadomienie
\end{itemize}
\subsection*{Dokumentacja programistyczna}
\subsubsection*{Dobrane technologie}
\begin{itemize}
    \item Angular + NativeScript
    \item Dane są przechowywane Google Firebase
    \item Powiadomienia push są obsługiwane przez Google Firebase Cloud Messaging
\end{itemize}
\end{document}