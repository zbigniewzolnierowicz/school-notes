\documentclass[a4paper]{article}
\usepackage[T1]{fontenc}
\usepackage[spanish,english,polish]{babel}
\usepackage{graphicx}
\usepackage{booktabs}
\usepackage[normalem]{ulem}
\usepackage{float}
\usepackage{indentfirst}
\useunder{\uline}{\ul}{}
\usepackage[utf8]{inputenc}
\usepackage{minted}
\usemintedstyle{xcode}
\begin{document}
\title{{\huge Lekcja 07} \\
    {\Large Zmienne}}
\author{Zbigniew Żołnierowicz}
\date{04.11.2019}
\maketitle
\section{Zasięg zmiennych}
\mintinline{javascript}{let} i \mintinline{javascript}{const} jest w zakresie blokowym zmiennych (i zakresie funkcji), \mintinline{javascript}{var} jest w zakresie funkcji bez zakresu blokowego.
\begin{minted}{javascript}
    var a = "a - zakres globalny";
    let b = "b - zakres globalny";
    const c = "c - zakres globalny";
    // Blok: {
    var a = "a - zakres lokalny";
    let b = "b - zakres lokalny";
    const c = "c - zakres lokalny";
    console.log(a); // "a - zakres lokalny"
    console.log(b); // "b - zakres lokalny"
    console.log(c); // "c - zakres lokalny"
    // }
    console.log(a); // "a - zakres lokalny"
    console.log(b); // "b - zakres globalny"
    console.log(c); // "c - zakres globalny"
\end{minted}
\subsection{Deklaracja funkcji z \mintinline{javascript}{var}}
\begin{minted}{javascript}
    function test() {
        var a = true;
        if (true) {
            var a = false;
            console.log(a);
        }
        console.log(a);
    }
    test();
    /*
        false
        false
    */
\end{minted}
\pagebreak
\subsection{Deklaracja funkcji z \mintinline{javascript}{let}}
\begin{minted}{javascript}
    function test() {
        let a = true;
        if (true) {
            let a = false;
            console.log(a);
        }
        console.log(a);
    }
    test();
    /*
        false
        true
    */
\end{minted}
\subsection{Porównanie \mintinline{javascript}{var} i \mintinline{javascript}{let}}
\begin{minted}{javascript}
    var i = 5;
    for (let i = 0; i < 10; i++) {
        console.log(i);
    }
    console.log(i);
    /*
    0
    1
    2
    3
    4
    5
    6
    7
    8
    9
    5
    */
\end{minted}
\begin{minted}{javascript}
    let i = 5;
    for (let i = 0; i < 10; i++) {
        console.log(i);
    }
    console.log(i);
    /*
    0
    1
    2
    3
    4
    5
    6
    7
    8
    9
    5
    */
\end{minted}
\end{document}