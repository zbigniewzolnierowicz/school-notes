\documentclass[a4paper]{article}
\usepackage{polski}
\usepackage[utf8]{inputenc}
\usepackage{datetime}
\usepackage{tikz}
\begin{document}
\title{
        {\huge Lekcja 34} \\
        {\large Powtórzenie wiadomości o Młodej Polsce}
}
\author{Zbigniew Żołnierowicz}
\date{29.04.2019}
\maketitle
\section{Nazwy Młodej Polski}
\begin{itemize}
    \item[Neoromantyzm] Odniesienia do romantyzmu
    \item[Modernizm] manifest nowoczesności, wynalezienie aparatu fotograficznego, początki kina
    \item[Secesja] 
    \item[Dekadentyzm] 
    \item[Fin de cicle] koniec wieku XIX
\end{itemize}
\section{Daty Młodej Polski}
Od 1890 do ok.\@ 1918
\section{Filozofia}
\subsection{Schopenhaueryzm}
Twórca: Artur Schopenhauer \\
Nirwana \\
totalny pesymizm, nie możemy pokonać cierpienia
\subsection{Nietzscheanizm}
Człowiek i nadczłowiek \\
Nadczłowieka nie obowiązuje kodeks moralny \\
Nadczłowiek traktuje innych jako pożywkę, żeby podbudować własną siłę
\subsection{Bergsonizm}
Henri Bergson

\begin{description}
    \item[Elun Vital] Pęd życia
\end{description}

Kierować się intuicją, siłą wewnętrzną która jest w nas
\section{Kierunki}
\subsection{Impresjonizm}
Próba uchwycenia chwili \\
Synestezja - odwoływanie się do różnych zmysłów
\subsection{Symbolizm}
Jan Kasprowicz - Krzak dzikiej róży \\
Przypomnieć symbol limby i róży
\subsection{Katastrofizm}
Jan Kasprowicz - Dzień Gniewu
\end{document}