\documentclass[a4paper]{article}
\usepackage{polski}
\usepackage[utf8]{inputenc}
\usepackage{datetime}
\begin{document}
\title{
        {\huge Lekcja 32} \\
        {\large W jaki sposób Kasprowicz odwołuje się do franciszkanizmu?}
}
\author{Zbigniew Żołnierowicz}
\date{01.04.2019}
\maketitle
\begin{itemize}
    \item Tytuł: Księga ubogich
    \item Brak negatywów, akceptuje wszystko co w życiu
    \item Stawia się w opozycji do materialistów
    \item Zmarł mu przyjaciel, ale przyrównuje to do naturalnego porządku życia (twierdzi też że śmierć jest lepsza niż życie)
    \item Postawa afirmacyjna (cieszy się z życia)
    \item Kontrast z Dniem Gniewu
\end{itemize}
\end{document}