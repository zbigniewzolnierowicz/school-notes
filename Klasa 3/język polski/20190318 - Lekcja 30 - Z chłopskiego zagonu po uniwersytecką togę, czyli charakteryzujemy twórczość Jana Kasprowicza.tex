\documentclass[a4paper]{article}
\usepackage{polski}
\usepackage[utf8]{inputenc}
\usepackage{datetime}
\begin{document}
\title{Lekcja 30 --- Z chłopskiego zagonu po uniwersytecką togę, czyli charakteryzujemy twórczość Jana Kasprowicza}
\author{Zbigniew Żołnierowicz}
\date{18.03.2019}
\maketitle
\section{Etapy twórczości Kasprowicza}
\subsection{Etap naturalistyczny: sonety ``W Chałupie''}
\subsubsection{Teza}
Tekst ``W Chałupie'' Kasprowicza to tekst naturalistyczny.
\subsubsection{Argumenty}
Naturalizm - pokazanie ``ciemnych'' stron życia: biedy, etc., ma budzić nasze obrzydzenie. (przykład: obraz Powiśla w Lalce)
\begin{itemize}
    \item Negatywne epitety
    \item Chałupa ma pełno dziur
\end{itemize}
\subsection{Etap modernistyczny: ``Krzak dzikiej róży''}
\subsubsection{Zadanie domowe}
Sonet 2
\paragraph{Wypisz elementy krajobrazu górskiego}
\begin{itemize}
    \item Niebieski kryształ
    \item Granit
    \item Ciemnosmreczyński las
    \item Wodospad Siklawa
    \item Mgła
\end{itemize}
\paragraph{Określ, jaki jest nastrój utworu}
Melanholijny, tajemniczy, refleksyjny, cisza przed burzą
\paragraph{Wypisz środki stylistyczne zastosowane przez poetę i określ ich funkcję (synestezja, synestezyjność)}
\subparagraph{Epitet}
Mają za zadanie uchwycenie kolorów i wrażeń emocjonalnych. Podział na pory dnia --- impresjonizm
\begin{itemize}
    \item ``niebieskim krysztale''
    \item ``wiewne fale''
    \item ``pas srebrnolity''
    \item ``szumna siklawa''
    \item ``skrytych załomach''
    \item ``cichym schronie''
\end{itemize}
\subparagraph{Neologizm}
\begin{itemize}
    \item ``głaźne''
    \item ``wiewne''
    \item ``zatopion''
\end{itemize}
\subparagraph{Personifikacja}
\begin{itemize}
    \item ``a przez mgły idą, przez błękity''
\end{itemize}
\subparagraph{Powtórzenie}
\begin{itemize}
    \item ``Jakby wzdychania, jakby żale''
    \item ``Jak lęk, jak żal, jak dech tęsknoty''
\end{itemize}
\subparagraph{Porównanie}
\begin{itemize}
    \item ``Jakby wzdychania, jakby żale''
    \item ``Jak lęk, jak żal, jak dech tęsknoty''
\end{itemize}
\subparagraph{Wielokropki}
\subparagraph{Synestezyjność}
Oddziaływanie na wiele zmysłów (wzrok, dotyk, słuch)

\paragraph{Wypisz znaczenia symbolu limby i krzaku dzikiej róży}
\begin{description}
    \item[Limba] Normalnie, wysokie drzewa, takie jak sosna limba (a także jej kolor - zieleń) to symbol życia. W tym wierszu jednak, \\
to drzewo jest spleśniałe, co symbolizuje chorobę.
    \item[Krzak dzikiej róży] Krzak jest ukryty w załomie, co może oznaczać życie kryjące się przed cierpieniem.
\end{description}
\subsubsection{Wnioski}
Kasprowicz odwołuje się do impresjonizmu, symbolizmu, dekadentyzmu.
\paragraph{Impresjonizm}
Ulotny moment, próba uchwycenia chwili, jak pora dnia oddziałowuje na otaczający świat
\paragraph{Symbolizm}

\subsection{Etap ekspresjonistyczny: ``Hymny''}
\subsection{Etap franciszkański: ``Witajcie, kochane góry''}
\end{document}