\documentclass[a4paper]{article}
\usepackage{polski}
\usepackage[utf8]{inputenc}
\usepackage{datetime}
\usepackage{tikz}
\begin{document}
\title{
        {\huge Lekcja 35} \\
        {\large Po co Bóg ``potworzył'' Dusiołki?}
}
\author{Zbigniew Żołnierowicz}
\date{03.06.2019}
\section{Napisz trzypunktowy plan, zawierający trzy najważniejsze wydarzenia utworu (stosuj zdania albo równoważniki zdań)}
\begin{description}
    \item[Odpoczynek] Bajdała znajduje wygodne miejsce do spania, zasypia.
    \item[Sen] We śnie atakuje go Dusiołek
    \item[Przebudzenie] Po przebudzeniu Bajdała obrzuca wołu i konia obelgami, że mu nie pomogły.
\end{description}
\section{Uzupełnij imiona informacjami dotyczącymi wyglądu i cech bohaterów wiersza. Przeprowadź analizę słowotwórczą słów}
\subsection{Bajdała}
Chłop, wąsaty
\subsection{Dusiołek}
półbabek, pysk z żabia ślimaty, zad tyli co kwoka
\section{Zapisz pytania do Boga}
\begin{itemize}
    \item ``Panie Boże, dlaczego stworzyłeś świat idealny, a nadal jest tyle zła?''
    \item ``Panie Boże, dlaczego stworzyłeś tylu ludzi, a nikt nie pomaga tym w cierpieniu?''
    \item ``Panie Boże, dlaczego akurat na mnie nasłałeś Dusiołka?''
\end{itemize}
\section{Objaśnij sens przenośny następujących elementów wiersza}
\begin{description}
    \item[Bajdała] Typowy, normalny człowiek
    \item[Dusiołek] ``Zło'' na świecie
    \item[Wędrówka Bajdały] Życie człowieka
    \item[Walka z Dusiołkiem] Wewnętrzna walka sumienia człowieka, ogólne zło na świecie
\end{description}
\section{Powtórzenie pojęć i terminów}
\begin{description}
    \item[teodycea] Jak pogodzić istnienie Boga ze złem na świecie
    \item[groteska] 
    \item[ballada] 
    \item[ballada filozoficzna] 
    \item[neologizm] 
    \item[synkretyzm] Połączenie różnych gatunków literackich
    \item[moralitet] Wybór bohatera pomiędzy dobrem a złem
    \item[homo viator] Motyw człowieka podróżnika, gdzie podróż to metafora życia
\end{description}
\section{Napisz, co jest twoim Dusiołkiem?}
Moim Dusiołkiem jest niemożność opanowania i skonsolidowania moich pomysłów tak, żeby były jedną całością, zamiast różnymi częściami. Dekoncentruję się, i obwiniam się za to resztę świata.
\end{document}