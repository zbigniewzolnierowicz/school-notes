\documentclass[a4paper]{article}
\usepackage{polski}
\usepackage[utf8]{inputenc}
\usepackage{datetime}
\begin{document}
\title{
        {\huge Lekcja 31} \\
        {\large Jakie oblicze Boga wyłania się z ekspresjonistycznej wizji Jana Kasprowcza?}
}
\author{Zbigniew Żołnierowicz}
\date{29.03.2019}
\maketitle
\section{Słowniczek pojęć związanych z hymnami Kasprowicza}
    \paragraph*{hymn}
    Utwór o charakterze podniosłym, pochwalnym, z podmiotem lirycznym zbiorowym
    \paragraph*{katastrofizm}
    \paragraph*{apokalipsa}
    Określenie z Biblii (Apokalipsa św. Jana) oznaczające koniec świata
    \paragraph*{oniryzm}
    \paragraph*{rytm śpiewny ``Bóg się rodzi'' Franciszka Karpińskiego  (``trąba dziwny dźwięk rozsieje, ogień skrzepnie, blask ściemnieje'}
    \paragraph*{wiersz wolny}
    \paragraph*{ekspresjonizm (portyka krzyku)}
    \paragraph*{prometeizm}
    \paragraph*{stylizacja biblijn}
    \paragraph*{nastrój grozy, bólu, zwątpienia}
    \paragraph*{świat w chaosie, rozpadzie, katastrofie}
    \paragraph*{człowiek nie jest umiłowaną przez Boga istotą, ale bezbronną ofiarą jego potęg}
    \paragraph*{groźny Stwórca}
    \paragraph*{wizja Sądu Ostatecznego nie napawa nadzieją na spotkanie Ojca, lecz przerażeniem przed Jego okrutną karą}
    \paragraph*{Bóg Starego Testamentu wzbudza w człowieku lęk, jest niedostępny, gniewny, a nawet okrutny; bezwzględny tyran}
    \paragraph*{hiperbole}
    \paragraph*{kontrasty: modlitwa i bluźnierstwo; dobro i zło; jasność i ciemność}
    \paragraph*{anafory, apostrofy, wyliczenia, eksklamacje (zawołania)}
    \paragraph*{kolory: czerń, czerwień, dla których tłem jest szarość pustki i ``próchna''}
    \paragraph*{perspektywa kosmiczna}
    \paragraph*{motyw tańca śmierci}
    \paragraph*{świat to ``Boże igrzysko''}
    \paragraph*{teodycea}
    \paragraph*{deizm}
\section{Wykorzystując słowa ze słowniczka, ułóż 15 zdań interpretujących ``Dzień gniewu'' Jana Kasprowicza}
Dzień Gniewu to subwersja konwencji hymnu.

Zamiast wielbienia Boga, sprawia, że się go boimy, poprzez pokazanie świata apokaliptycznego.

W samym wierszu występuje oniryzm.

Pierwsza zwrotka jest antytezą do "Bóg Się Rodzi", kolędy która pochwala narodziny Jezusa, a tutaj powoduje lęk, przez porównanie do trąb sygnalizujących koniec świata.

Wiersz jest ekspresjonistyczny, przez użycie licznych wykrzykników, które są cechą szczególną poetyki krzyku.

Przez bycie otoczonym przez pioruny, Bóg wydaje się groźny.

Zastosowanie licznych hiperboli nadaje wierszu groźnego nastroju.

Zabieg ten sprawia, że wiersz jest swojego rodzaju przestrogą.

Przestroga ta ma za zadanie powiedzieć grzesznikom, że muszą się poprawić, albo stawić się przed gniewem Boga.

Śmierć jest przedstawiona jako kobieta, Ewa, z wężem, który w Genesis dał jej zakazany owoc.

Liczne negatywne epitety i odniesienia do krwi mają za zadanie przestraszyć odbiorcę.

Epitety i odniesienia te nadają także nastrój bólu, ponieważ każdy człowiek ma w sobie krew.

Ze względu na bardzo szeroki zakres wierszu, mamy do czynienia z perspektywą kosmiczną, obejmującą cały świat.

Podmiot wini siebie za grzech pierwotny.

Ponieważ licznie stosowane są wyrażenia popularne w tekstach chrześcijańskich (amen, kyrie elejson), wiersz ma stylizację biblijną.
\end{document}
