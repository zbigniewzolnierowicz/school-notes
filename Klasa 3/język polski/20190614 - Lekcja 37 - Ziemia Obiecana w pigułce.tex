\documentclass{article}
\usepackage{polski}
\usepackage[utf8]{inputenc}
\usepackage{datetime}
\usepackage{tikz}
\usepackage{graphicx}
\begin{document}
\title{
        {\huge Lekcja 37} \\
        {\large ``Ziemia Obiecana'' w pigułce}
}
\author{Zbigniew Żołnierowicz}
\date{14.06.2019}
\maketitle
\section{Elementy świata przedstawionego}
\begin{description}
    \item[czas akcji] lata osiemdziesiąte XIX wieku
    \item[miejsce wydarzeń] Łódź
    \item[główni bohaterowie] Trzej bohaterowie, trzy narody
    \begin{itemize}
        \item Karol Borowiecki (Polak)
        \item Moryc Welt (Żyd)
        \item Maks Baum (Niemiec)
    \end{itemize}
\end{description}
\section{Najważniejsze sceny utworu}
\subsection{scena zakładania fabryki}
\subsection{scena śmierci Kesslera}
\subsection{scena strzelania do robotników}
\subsection{przejazd Anki ze wsi do miasta}
\section{Zagadnienia do interpretacji}
\subsection{antagonizmy narodowościowe}
\subsection{zróżnicowanie społeczne: ``wysadzeni z siodła'', ``wolni najmici''}
\subsection{naturalizm}
\subsection{fin de si\'ete - ``koniec wieku''}
\subsection{modernizm}
\subsection{kapitalizm}
\subsection{antyurbanizm}
\subsection{funkcja tytułu}
\subsection{język postaci: stylizacja środowiskowa}
\section{Motywy}
\subsection{miasto (bohater zbiorowy)}
\subsection{apokaliptyczna wizja miasta}
\subsection{siła pieniądza}
\subsection{karierowiczostwo, konformizm}
\subsection{przyjaźń}
\subsection{ziemia obiecana}
\subsection{piekło}
\subsection{moty biblijne}
\pagebreak
\section{Wybrane cytaty}
\begin{itemize}
    \item \begin{quote}
        Zakładamy fabrykę. Ja nie mam nic, ty nie masz nic, on nie ma nic... To razem właśnie mamy tyle, w sam raz, żeby założyć wielką fabrykę.
    \end{quote}
    \item[Karol] \begin{quote}
        Jest pan maszyną nie człowiekiem.    
    \end{quote}
    \item[Karol] \begin{quote}
        Tyle materiału na nic
    \end{quote}
    \item \begin{quote}
        Uczciwi mają dobre niebo, po co im dobry czas.
    \end{quote}
    \item[Maks] \begin{quote}
        Mój ojciec nie ma nic do stracenia oprócz honoru, a tym towarem Łódź nie handluje.
    \end{quote}
    \item[Trawiński] \begin{quote}
        Dla mądrych to jest dobry czas, kiedy będzie dla uczciwych?
    \end{quote}
\end{itemize}
\end{document}