\documentclass{article}
\usepackage[T1]{fontenc}
\usepackage[polish]{babel}
\usepackage{inconsolata}
\usepackage{indentfirst}
\usepackage{minted}
\begin{document}
\title{{\huge Lekcja 22} \\
{\large Perspektywy}}
\author{Zbigniew Żołnierowicz}
\date{29.03.2019}
\maketitle
Perspektywa \emph{(ang. views)} jest strukturą logiczną, udostępniającą wybrane informacje przechowywane w relacjach bazy danych.

\section{Własności}

\begin{itemize}
    \item Definiowana w oparciu o relacje (relacje bazowe) lub inne perspektywy (perspektywy bazowe)
    \item Nie posiada własnych danych, nie jest materializowana na dysku
    \item Przechowywana w postaci zapytania
\end{itemize}

\section{Cele stosowania}

\begin{itemize}
    \item Ograniczenie dostępu do danych --- zabezpieczenie przed nieautoryzowanym dostępem
    \item Uproszczenie schematu bazy danych, uproszczenie zapytań
    \item Uniezależnienie aplikacji od zmian w strukturze bazy danych
    \item Prezentowanie danych w inny sposób niż dane w relacjach i perspektywach bazowych (m.in.\@ zmiana naz atrybutów, formatów danych, itp.)
    \item Dodatkowa kontrola poprawności wprowadzanych danych
\end{itemize}

Perspektywy służą do dostosowania bazy danych do potrzeb róznych grup użytkowników.
Stanowią widok, perspektywę, z jakiej dana klasa użytkowników widzi bazę danych.
Różne grupy użytkowników mogą mieć odmienne ``perspektywy'' na dane w bazie danych.
\pagebreak

(nawiasy kwadratowe oznaczają opcjonalne wartości)

\begin{minted}{sql}
    CREATE VIEW nazwa_perspektywy [ (kolumna1, kolumna2, ...) ]
    AS
    SELECT zapytanie definiujące perspektywę;
\end{minted}

Jeżeli nie podamy nazw kolumn, zostaną odziedziczone po zapytaniu \mintinline{sql}{SELECT}.

\begin{minted}{sql}
    CREATE VIEW prac_zesp_30 (id, nazwisko, pensja)
    AS
    SELECT id, nazwisko, wynagrodzenie
    FROM pracownicy WHERE NrZespolu = 30;
\end{minted}

\section{Ograniczenia}

Każdy system baz danych ogranicza zakres perspektyw, przez które można dokonywać zmian w bazie danych. W odniesieniu do standardu mamy do czynienia z następującymi ograniczeniami:

\begin{itemize}
    \item W klauzuli \mintinline{sql}{SELECT} nie może być słowa kluczowego \mintinline{sql}{DISTINCT}.
    \item W klauzuli \mintinline{sql}{FROM} może być tylko jedna nazwa tabeli lub jedna nazwa perspektywy --- spełniająca definiowane kryteria
    \item Na liście \mintinline{sql}{SELECT} mogą znajdować się tylko nazwy kolumn.
    \item W klauzuli \mintinline{sql}{WHERE} nie może być podzapytania
    \item W zapytaniu nie mogą występować klauzule \mintinline{sql}{GROUP BY} i \mintinline{sql}{HAVING}.
\end{itemize}

Powyższe warunki zapewniają to, aby jednoznacznie był określony wiersz w tabeli, której dotyczy zmiana.

\section{Rodzaje perspektyw}

\subsection{Perspektywy proste}

Perspektywy proste oparte są na jednej relacji bazowej i nie zawierają:

\begin{itemize}
    \item operatorów zbiorowych
    \item operatora \mintinline{sql}{DISTINCT}
    \item funkcji grupowych i analitycznych
    \item grupowania
    \item sortowania
    \item podzapytań w klauzuli \mintinline{sql}{SELECT}
\end{itemize}

Perspektywy proste mogą służyć do wstawiania, modyfikowania i usuwania krotek z relacji bazowej.

\subsection{Perspektywy złożone}

Perspektywy złożone są oparte na wielu relacjach i perspektywach bazowych i wykorzystują operatory zbiorowe, funkcje, grupowanie, sortowanie, połąćzenia, funkcje analityczne, itd.

Perspektywy złożone służą tylko i wyłącznie do odczytu.

\begin{minted}{sql}
    CREATE VIEW kobiety(id, nazwisko, wynagrodzenie)
    AS
    SELECT id, nazwisko, wynagrodzenie
    FROM pracownicy
    WHERE plec = 'K';
\end{minted}

\begin{minted}{sql}
    CREATE VIEW place_plec(plec, srednia, maksymalna, liczba)
    AS
    SELECT plec, AVG(wynagrodzenie), MAX(wynagrodzenie), COUNT(id)
    FROM pracownicy
    GROUP BY plec
    ORDER BY MAX(wynagrodzenie) DESC;
\end{minted}

\section{Tworzenie perspektywy z opcją sprawdzania}

W celu zapewnienia ograniczenia danych operowanych przy użyciu perspektywy do danych określonych przez jej warunek \mintinline{sql}{WHERE}, użyteczna jest dodatkowa opcja definicji perspektywy:

\begin{minted}{sql}
    CREATE VIEW nazwa_perspektywy [ (kolumna1, kolumna2, ...) ]
    AS
    SELECT zapytanie definiujące perspektywę
    WITH CHECK OPTION;
\end{minted}

Jeżeli została użyta opcja \mintinline{sql}{WITH CHECK OPTION}, to przy wykonywaniu \mintinline{sql}{INSERT} i \mintinline{sql}{UPDATE} następuje sprawdzenie, czy wstawiany bądź modyfikowany wiersz spełnia warunek określony w klauzuli \mintinline{sql}{WHERE}. Jeśli spełnia, operacja jest wykonywana. Jeśli nie, to operacja nie jest wykonywana.
Gwarantuje to dodatkową ochronę danych zapewniając, że zmiany dokonywane przez użytkowników przez perspektywę są wyłącznie ograniczone do zbioru danych, do których oglądania i modyfikowania jest uprawniony dany użytkownik.

\pagebreak

\section{Usuwanie perspektyw}

Każdą perspektywę można usunąć.

\begin{minted}{sql}
    DROP VIEW nazwa_perspektywy;
\end{minted}

Usunięcie perspektywy nie skutkuje usunięciem rekordów z tabeli bazowej. Znika jedynie wirtualny widok, który wcześniej został utworzony. Wydanie polecenia usuwającego tabelę bazową spowoduje, że perspektywa ta staje się niedostępna.

\end{document}