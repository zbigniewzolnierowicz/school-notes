\documentclass[a4paper]{article}
\usepackage[T1]{fontenc}
\usepackage[polish]{babel}
\usepackage{inconsolata}
\usepackage{minted}
\begin{document}
\title{Lekcja 21 --- DDL --- Kontynuacja}
\author{Zbigniew Żołnierowicz}
\date{20.03.2018}
\maketitle
\section{Klauzula {\tt DEFAULT}}
Gdy system nie otrzyma bezpośrednio podanej wartości dla kolumny, szuka klauzuli {\tt DEFAULT} i stosuje określoną w niej wartość. Może być ona: 
\begin{itemize}
  \item Literałem odpowiedniego typu danych
  \item Wartością zwracaną przez system (np. bieżący czas, aktualną datę)
\end{itemize}
Jeżeli nie podamy klauzuli {\tt DEFAULT}, a kolumna akceptuje wartości {\tt NULL}, system jako wartość domyślną zastosuje {\tt NULL}. Jeżeli także to nie jest możliwe, pojawi się komunikat o błędzie.
\section{Ograniczenie {\tt CHECK}}
Ograniczenie {\tt CHECK} sprawdza, czy wiersze tabeli spełniają wyrażenie logiczne i odrzuca te wiersze, dla których warunek ma wartość {\tt FALSE}. Gdy warunek ma wartość {\tt TRUE} lub {\tt UNKNOWN}, wiersz jest akceptowany.

Za pomocą ograniczenia {\tt CHECK} często sprawdza się, czy wartość należy do jakiegoś przedziału (np.\mintinline{sql}{CHECK (rabat BETWEEN 1 AND 10)}) lub czy wartość kolumny należy do określonego zbioru (np. \mintinline{sql}{CHECK(płeć IN 0,1,9)})
\section{Klauzula {\tt REFERENCES}}
Ograniczenie referencyjne - definiuje zależność między relacjami. Klucz obcy w relacji musi wskazywać na klucz podstawowy lub unikalny w {\tt wskazywanej\_tabeli}.
Klucz obcy dopuszcza wartości puste, chyba że zostaną wyeliminowane przez inne ograniczenia.

Składnia (dla jednego atrybutu)
\begin{minted}{sql}
  CREATE TABLE nazwa_tabeli (
  nazwa_kolumny typ_danych
  (CONSTRAINT nazwa) REFERENCES nazwa_wskazywanej_tabeli(wskazywana_kolumna)
  );
\end{minted}

\begin{minted}{sql}
  CREATE TABLE pracownicy (
  ...,
  NrZespolu int REFERENCES Zespoły(IdZesp),
  ...
  );
\end{minted}

Definicja klucza obcego z dwoma atrybutami.

Założenie: klucz podstawowy w tabeli wskazywanej {\tt POMIESZCZENIA} składa się z atrybutów o nazwach: symbol\_budynku i numer\_pomieszczenia.

\begin{minted}{sql}
  (
  budynek CHAR(10),
  nr_pomieszczenia DECIMAL(3),
  ...
  CONSTRAINT fk_pomieszczenia
  FOREIGN KEY(budynek, nr_pomieszczenia)
  REFERENCES pomieszczenia(symbol_budynku, numer_pomieszczenia)
  )
\end{minted}

\section{Operacje związane z referencjami}

\subsection{ON DELETE}

\subsection{\tt ON DELETE CASCADE}

Razem z wierszem tabeli nadrzędnej zostają usunięte wszystkie wiersze tabeli podrzędnej, w których wartości klucza obcego wskazują na usuwany wiersz (pod warunkiem, że usunięcie tych wierszy jest możliwe bez naruszenia innych więzów referencyjnych).

\subsection{\tt ON DELETE SET NULL}

Przy usuwaniu wiersza, do którego są odwołania przez wartości kluczy obcych, następuje wstawienie wartości \mintinline{sql}{NULL} jako wartości klucza obcego.

\subsection{\tt ON DELETE SET DEFAULT}

Przy usuwaniu wiersza, do którego są owołania przez wartości kluczy obcych, następuje wstawienie wartości domyślnej jako kolumny, jako wartości klucza obcego. Jeśli nie ma jawnej wartości domyślnej, \mintinline{sql}{NULL} staje się domyślną wartością domyślną kolumny.

\subsection{\tt ON DELETE NO ACTION}

Przy usuwaniu wiersza, do którego są odwołania przez wartości kluczy obcych, podnoszony jest błąd, a polecenie \mintinline{sql}{DELETE} jest wycofywane. Jest to domyślne ustawienie.

\subsection{\tt ON UPDATE CASCADE}

Razem z wierszem tabeli nadrzędnej zostają zaktualizowane wszystkie wiersze tabeli podrzędnej.

\subsection{\tt OP UPDATE SET NULL}

Po modyfikacji wiersza wstawiany jest \mintinline{sql}{NULL}.

\subsection{\tt ON UPDATE SET DEFAULT}

Po modyfikacji wiersza wstawiany jest \mintinline{sql}{DEFAULT}.

\subsection{\tt ON UPDATE NO ACTION}

Po modyfikacji wiersza pojawia się błąd.

\section{Autonumeracja}

\begin{minted}{sql}
  IDENTITY(seed, increment)
\end{minted}

Gdzie {\tt seed} to pierwsza wartość wstawiona do kolumny, a {\tt increment} to wartość przyrostu kolejnej wartości.
Jeżeli {\tt seed} i {\tt increment} nie zostały zadeklarowane, przyjmą domyślne wartości (1,1).

\begin{minted}{sql}
  CREATE TABLE (
  id_zespołu IDENTITY(1,1) PRIMARY KEY
  );
\end{minted}

\section{Przykłady}

\subsection{Dodawanie nowej kolumny do istniejącej tabeli}

\begin{minted}{sql}
  ALTER TABLE nazwa_tabeli
  ADD nazwa_kolumny typ_danych DEFAULT wartość_domyślna ograniczenia;
\end{minted}

\subsection{Zmiana typu danych kolumny w istniejącej tabeli}

\begin{minted}{sql}
  ALTER TABLE nazwa_tabeli
  ALTER COLUMN nazwa_kolumny typ_danych;
\end{minted}

\subsection{Usunięcie kolumny z istniejącej tabeli}

\begin{minted}{sql}
  ALTER TABLE nazwa_tabeli
  DROP COLUMN nazwa_kolumny;
\end{minted}

\section{Zmiana więzów spójności}

Zmiana więzów spójności zawze sprowadza się do modyfikacji tabeli, w której więzy istnieją,
lub mają zostać dodane. Nie istnieje możliwość użycia instrukcji \mintinline{sql}{CREATE},
\mintinline{sql}{DROP}, \mintinline{sql}{DROP} w zastosowaniu bezpośrednio do węzłów.

\begin{minted}{sql}
  ALTER TABLE nazwa_tabeli
  DROP CONSTRAINT nazwa_ograniczenia;
\end{minted}

\section{Zadania}

\subsection{Utwórz tabelę Szkoły o następujących kolumnach i ograniczeniach}

\begin{minted}{sql}
  CREATE TABLE Szkoły (
  Id NUMERIC(4),
  Nazwa VARCHAR(100),
  Adres VARCHAR(100),
  Budzet NUMERIC(10,2),
  Zalozona DATE,
  CONSTRAINT pk_Id PRIMARY KEY(Id),
  CONSTRAINT un_Nazwa UNIQUE(Nazwa)
  ); 
\end{minted}
\subsection{Utwórz tabelę Projekty o następujących kolumnach i ograniczeniach}

\begin{minted}{sql}
  CREATE TABLE Projekty (
  Id NUMERIC(4),
  Opis VARCHAR(20)
  CONSTRAINT un_Opis UNIQUE NOT NULL,
  Data_rozpoczecia DATE DEFAULT GETDATE(),
  Data_zakonczenia DATE,
  CHECK(Data_zakonczenia > Data_rozpoczecia),
  Fundusz VARCHAR(7,2),
  CONSTRAINT pk_Id PRIMARY KEY(Id),
  );
\end{minted}

\subsection{Utwórz tabelę Pracownicy, Miasta, DawniPracownicy, Zespoły z kluczami obcymi}

\begin{minted}{sql}
  CREATE TABLE Miasta (
    Id NUMERIC(4) IDENTITY(1,1),
    Nazwa VARCHAR(20),
    CONSTRAINT pk_Id PRIMARY KEY(Id),
    CONSTRAINT nn_Nazwa NOT NULL(Nazwa)
  );
\end{minted}

\begin{minted}{sql}
  CREATE TABLE Zespoły (
    Id NUMERIC(4) IDENTITY(1,1),
    Nazwa VARCHAR(20),
    Kierownik NUMERIC(4),
    Miasto NUMERIC(4),
    CONSTRAINT pk_Id PRIMARY KEY(Id),
    CONSTRAINT nn_Nazwa NOT NULL(Nazwa),
    CONSTRAINT fk_Miasto_MiastaId FOREIGN KEY (Miasto)
    REFERENCES Miasta(Id)
  );
\end{minted}

\begin{minted}{sql}
  CREATE TABLE DawniPracownicy (
    Id Numeric(4),
    Imię VARCHAR(20),
    DrugieImię VARCHAR(20),
    Nazwisko VARCHAR(20),
    Płeć CHAR(1) CHECK IN("K", "M", NULL)
    Wynagrodzenie NUMERIC(4),
    NrZespołu NUMERIC(4),
    CONSTRAINT pk_Id PRIMARY KEY(Id),
    CONSTRAINT nn_Imię NOT NULL(Imię),
    CONSTRAINT nn_Nazwisko NOT NULL(Nazwisko),
    CONSTRAINT fk_NrZespołu_ZespołyId FOREIGN KEY (NrZespołu)
    REFERENCES Zespoły(Id)
  ); 
\end{minted}

\begin{minted}{sql}
  CREATE TABLE Pracownicy (
    Id Numeric(4),
    Imię VARCHAR(20),
    DrugieImię VARCHAR(20),
    Nazwisko VARCHAR(20),
    Płeć CHAR(1) CHECK IN("K", "M", NULL)
    Wynagrodzenie NUMERIC(4),
    NrZespołu NUMERIC(4),
    Aktywny NUMERIC(1) CHECK IN(0, 1) DEFAULT 0,
    CONSTRAINT pk_Id PRIMARY KEY(Id),
    CONSTRAINT nn_Imię NOT NULL(Imię),
    CONSTRAINT nn_Nazwisko NOT NULL(Nazwisko),
    CONSTRAINT nn_Aktywny NOT NULL(Aktywny),
    CONSTRAINT fk_NrZespołu_ZespołyId FOREIGN KEY (NrZespołu)
    REFERENCES Zespoły(Id)
  ); 
\end{minted}
\pagebreak
\begin{minted}{sql}
  ALTER TABLE Zespoły
  CREATE CONSTRAINT fk_Kierownik_PracownicyId FOREIGN KEY (Kierownik)
  REFERENCES Pracownicy(Id);
\end{minted}
\subsection{Wprowadź nową osobę zatrudnioną dzisiaj. Jan Kowalski, zarabiający 100 zł.}
\begin{minted}{sql}
  INSERT INTO Pracownicy (Imię, Nazwisko, Wynagrodzenie)
  VALUES ('Jan', 'Kowalski', 100);
\end{minted}
\subsection{Przypisz domyślną wartość 3 dla identyfikatora miasta w zespołach}
\begin{minted}{sql}
  ALTER TABLE Zespoły
  ALTER COLUMN Miasto DEFAULT 3;
\end{minted}
\subsection{Usuń klucz obcy z tabeli Zespoły}
\begin{minted}{sql}
  ALTER TABLE Zespoły
  DROP CONSTRAINT fk_Miasto_MiastaId;
\end{minted}
\end{document}