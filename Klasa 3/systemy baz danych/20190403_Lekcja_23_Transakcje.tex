\documentclass{article}
\usepackage[T1]{fontenc}
\usepackage[polish]{babel}
\usepackage{inconsolata}
\usepackage{indentfirst}
\usepackage{booktabs}
\usepackage{minted}
\usepackage{graphicx}
\begin{document}
\title{{\huge Lekcja 23} \\
{\large Transakcje}}
\author{Zbigniew Żołnierowicz}
\date{29.03.2019}
\maketitle
\section{Transakcja}
Transakcja jest sekwencją logicznie powiązanych operacji na bazie danych oraz obiektach rzeczywistych,
która przeprowadza bazę danych z jednego stanu spójnego w inny stan spójny.
O transakcjach mówimy w kontekście współużytkowania danych przez wielu użytkowników i powstawania ewentualnych wzajemnych konfliktów
Samo czytanie danych (bez ich modyfikacji) nie powoduje konfliktów.
Przyczyną powstawania konfliktów są modyfikacje danych w środowisku wielu użytkowników.
System zarządzania bazą danych powinien rozwiązać potencjalne konflikty powstające przy przetwarzaniu transakcji.
\section{Wymagania stawiane transakcjom}
\Large{\textbf{ACID} - \textbf{A}tomicity, \textbf{C}onsistency, \textbf{I}solation, \textbf{D}urability.}
\normalsize
\begin{description}
    \item[Niepodzielność {\em (Atomicity)}] Albo wszystkie modyfikacje wchodzące w skład transakcji są wykonywane, albo żadna.
    \item[Spójność {\em (Consistency)}] Wszystkie transakcje muszą zachować spójność i integralność bazy danych.
    \item[Izolacja {\em (Isolation)}] Jeśli transakcja modyfikuje dane, to te dane mogą być czasowo niespójne, dlatego muszą być niedostępne dla innych transakcji dopóty, dopóki transakcja nie skończy ich używać.
    \item[Trwałość {\em (Durability)}] Gdy transakcja się kończy, to wszystkie zmiany przez nią dokonane muszą zostać w pełni utrwalone - nawet w przypadku awarii sprzętu.
\end{description}
\pagebreak
\section{Właściwości transakcji}
Transakcja to jedna modyfikacja lub seria modyfikacji bazy danych traktowana tak jakby była to jedna modyfikacja.
Jeśli modyfikacje zachodzą, to wszystkie zachodzą w jednej chwili.
Użytkownicy mogą zobaczyć stan bazy danych przed transakcją lub po jej zakończeniu, ale nigdy w trakcie.
\section{Przykłady transakcji i zagrożeń}
\begin{itemize}
    \item Rezerwacja miejsca w samolocie --- możliwe dwukrotne sprzedanie tego samego miejsca
    \item Zwiększenie pensji 300 pracowników --- możliwa awaria sprzętu w trakcie zapisu na dysk
\end{itemize}
\section{Zatwierdzanie transakcji}
W języku {\tt SQL} do zatwierdzenia transakcji służy instrukcja {\tt COMMIT}.
Do odwołania transakcji służy polecenie {\tt ROLLBACK}.
W standardzie języka {\tt SQL} nie ma instrukcji zaczynającej transakcję.

SQL Server umożliwia przeprowadzanie transakcji w 3 trybach.
\subsection{\em AutoCommit}
Każde polecenie {\tt SQL} stanowi odrębną transakcję.
\subsection{\em Explicit}
Transakcję trzeba rozpocząć poleceniem \mintinline{sql}{BEGIN TRANSACTION} i zakończyć poleceniem \mintinline{sql}{COMMIT} lub \mintinline{sql}{ROLLBACK}.
\subsection{\em Implicit}
Transakcja rozpoczyna się automatycznie po wykonaniu 1 polecenia {\tt SQL} i trzeba ją zakończyć jawnie poleceniem \mintinline{sql}{COMMIT} lub \mintinline{sql}{ROLLBACK}.
\pagebreak
\section{Izolacja transakcji - język {\tt SQL}}
W języku SQL określono 4 poziomy izolacji transakcji.
W celu zdefiniowania poziomów izolacji transakcji określono 3 rodzaje nieprawidłowych odczytów.
Poziom izolacji transakcji określa dozwolone dla danego poziomu nieprawidłowe odczyty.
Im wyższy poziom izolacji, tym mniej dopuszczalnych nieprawidłowych odczytów.
Im wyższy poziom izolacji, tym wolniejsza praca systemu zarządzania bazą danych.
Dopuszczenie nieprawidłowychodczytów przyspiesza pracę systemu zarządzania bazą danych w porównaniu z najwyższym poziomem izolacji --- szeregowym.
\subsection{Nieprawidłowe odczyty}
\subsubsection{Brudny odczyt (\emph{Dirty Read})}
Pierwsza transakcja modyfikuje wiersz, a druga go czyta, zanim zmiana została zatwierdzona przez instrukcję \mintinline{sql}{COMMIT}. Jeśli pierwsza transakcja została anulowana, zmiana nie miała miejsca i druga transakcja przeczytała wiersz, który naprawdę nigdy nie istniał.
\subsubsection{Odczyt bez powtórzeń (\emph{Non-repeatable read})}
Pierwsza transakcja czyta wiersz. Druga go usuwa lub modyfikuje i wykonuje \mintinline{sql}{COMMIT} przed pierwszą. Teraz pierwsza tranakcja mogłaby przeczytać ten sam wiersz jeszcze raz i otrzymać inne wyniki.
\subsubsection{Odczyt widmo (\emph{Phantom})}
Pierwsza transakcja odczytuje wiersze spełniające predykat. Druga wstawia wartości (instrukcja \mintinline{sql}{INSERT}) lub je modyfikuje (instrukcja \mintinline{sql}{UPDATE}) tak, że one również spełniają predykat. Następne wykonanie tego samego zapytania przez pierwszą transakcję da inne wyniki.
\begin{table}
    \centering
    \resizebox{\textwidth}{!}{%
    \begin{tabular}{@{}llll@{}}
    \toprule
    \textbf{Poziom izolacji} & \textbf{Brudny odczyt} & \textbf{Odczyt bez powtórzeń} & \textbf{Odczyt widmo} \\ \midrule
    {\tt READ UNCOMMITED} & TAK & TAK & TAK \\
    {\tt READ COMMITED} & NIE & TAK & TAK \\
    {\tt REPEATABLE READ} & NIE & NIE & TAK \\
    {\tt SERIALIZABLE} & NIE & NIE & NIE \\ \bottomrule
    \end{tabular}%
    }
    \caption{Poziomy izolacji i dozwolone odczyty}
    \label{poziomy-izolacji}
    \end{table}
\end{document}