\documentclass[a4paper]{article}
\usepackage{polski}
\usepackage{minted}
\usepackage[utf8]{inputenc}
\usepackage{graphicx}
\usepackage{booktabs}
\begin{document}
\title{Lekcja 20 --- Stosowanie dat i czasu}
\author{Zbigniew Żołnierowicz}
\date{15.03.2019}
\maketitle
\section{Typy danych}
Wymagania dotyczące pamięci dla 3 ostatnich typów danych zależne są od użytej dokładności. Dokładność specyfikowana jest w postaci liczby całkowitej z zakresu od 0 do 7 i reprezentuje wielkość ułamkowej części sekundy (liczbę cyfr po przecinku). Jeśli dokładność nie została wyspecyfikowana domyślnie system zakłada wartość 7.

\begin{table}[]
\centering
\resizebox{\textwidth}{!}{%
\begin{tabular}{@{}lllll@{}}
\toprule
Typ & Pamięć & Zakres dat & Dokładność & Zalecany format \\ \midrule
datetime & 8 & Od 1 stycznia 1753 do 31 grudnia 9999 & 3 1/3 milisekund & \begin{tabular}[c]{@{}l@{}}`YYYYMMDD hh:mm:ss:nnn'\\ `20090212 12:30:15:123'\end{tabular} \\
smalldatetime & 4 & Od 1 stycznia 1900 do 6 czerwca 2079 & 1 minuta & \begin{tabular}[c]{@{}l@{}}`YYYYMMDD hh:mm'\\ `20090212 12:30'\end{tabular} \\
date & 3 & Od 1 stycznia 0001 do 31 grudnia 9999 & 1 dzień & \begin{tabular}[c]{@{}l@{}}`YYYY-MM-DD'\\ `2009--02--12'\end{tabular} \\
time & 3--5 & Nie dotyczy & 100 nanosekund & \begin{tabular}[c]{@{}l@{}}`hh:mm:ss:nnn'\\ `12:30:15:1234567'\end{tabular} \\
datetime2 & 6--8 & Od 1 stycznia 0001 do 31 grudnia 9999 & 100 nanosekund & \begin{tabular}[c]{@{}l@{}}`YYYY-MM-DD hh:mm:ss:nnnnnnnn'\\ `2009--02--12 12:30:15:1234567'\end{tabular} \\
datetimeoffset &  &  &  &  \\ \bottomrule
\end{tabular}%
}
\end{table}
\section{Literały}
System SQL Server nie zapewnia środków do przedstawienia literału daty i godziny, zamiast tego pozwala wyspecyfikować literał innego typu, który można przekształcić na typ danych daty i godziny.

Najlepszym rozwiązaniem jest stosowanie ciągu znaków do wyrażenia wartości daty i godziny według przykładu:
\begin{minted}{sql}
SELECT id, imie, nazwisko, zatrudnienie
FROM Pracownicy
WHERE zatrudnienie = '20070212';
\end{minted}

\section{Bierząca data i godzina}

\begin{minted}{sql}
SELECT
  GETDATE() AS [GETDATE]
  CURRENT_TIMESTAMP AS [CURRENT_TIMESTAMP]
  SYSDATETIME() AS [SYSDATETIME]
  SYSDATETIMEOFFSET() AS SYSDATETIMEOFFSET[];
\end{minted}
\begin{minted}{sql}
SELECT
  CAST (SYSDATETIME() AS DATE) AS [data],
  CAST (SYSDATETIME() AS TIME) AS [czas];
\end{minted}

\section{Funkcje daty i czasu}

{\tt YEAR} (data), {\tt MONTH} (data), {\tt DAY} (data) --- dokonują wyodrębnienia z daty, odpowiednio roku, miesiąca oraz dnia.

\begin{minted}{sql}
SELECT YEAR('2013-02-12') AS Rok,
       MONTH('2013-02-12') AS Miesiąc,
       DAY('2013-02-12') AS Dzień;
\end{minted}

{\tt DATEADD} (datepart, liczba, data) --- dodaje (lub odejmuje) lczbę jednostek daty/czasu określonych za pomocą DATEPART, np.\@ dni (day, dd, d), lat (years, yy, yyyy), miesięcy (month, mm, m), minut (minute, mi, m) itd.\@ od zadanej day.
{\tt DATEDIFF} (datepart, startdate, enddate) --- różnica pomiędzy dwiema datami (end-start) wyrażona w jednostkach określonych przez datepart. Wiek pracowników:
\begin{minted}{sql}
SELECT Imię, Nazwisko, DATEDIFF(yy, DataUrodzenia, GETDATE()) as Wiek
FROM Pracownicy;
\end{minted}

{\tt DATEPART} (datepart, data) --- wyciąga określoną parametrem datepart, jednostkę podanej daty.
\begin{minted}{sql}
SELECT DATEPART(yy, GETDATE()) as CurrentYear,
       DATEPART(mm, GETDATE()) as CurrentMonth,
       DATEPART(dd, GETDATE()) as CurrentDay;
\end{minted}
{\tt DATENAME} (datepart, data) --- podobna do datepart, zwraca wartość znakową określonej parametrem datepart, części daty w tym nazwę dnia tygodnia, miesiąca.

\begin{minted}{sql}
SELECT DATENAME(dw, GETDATE()) AS DzienTygodnia,
       DATENAME(mm, GETDATE()) AS Niesiac;
\end{minted}

\section{Przykładowe kwerendy}

\subsection{Wyświetl wszystkich pracowników, którzy zostali zatrudnieni po 12 grudnia 2009 roku.}

\begin{minted}{sql}
SELECT *
FROM Pracownicy
WHERE DataZatrudnienia > '2009-12-12';
\end{minted}

\subsection{Dla każdego pracownika wyświetl rok, miesiąc, dzień i dzień tygodnia jego urodzenia.}

\begin{minted}{sql}
SELECT Imie, Nazwisko,
DATEPART(yy, DataUrodzenia) AS RokUrodzenia,
DATEPART(mm, DataUrodzenia) AS MiesiacUrodzenia,
DATEPART(dd, DataUrodzenia) AS DzienUrodzenia,
DATENAME(dw, DataUrodzenia) AS DzienTygodniaUrodzenia
FROM Pracownicy;
\end{minted}

\subsection{Wyświetl wszystkich pracowników urodzonych w 1989 roku.}
\begin{minted}{sql}
SELECT *
FROM Pracownicy
WHERE YEAR(DataUrodzenia) = '1989';
\end{minted}

\subsection{Ile lat minęło od 1 stycznia 1990 roku?}

\begin{minted}{sql}
SELECT DATEDIFF(yy, '1990-01-01', GETDATE());
\end{minted}

\subsection{Dla każdego pracownika wyświetl imię, nazwisko, wiek oraz ile miesięcy pracuje.}

\begin{minted}{sql}
SELECT Imie, Nazwisko,
DATEDIFF(yy, DataUrodzenia, GETDATE()) AS Wiek,
DATEDIFF(mm, DataZatrudnienia, GETDATE()) AS MiesiecyPracy
FROM Pracownicy;
\end{minted}

\subsection{Wyświetl wszystkich pracowników, którzy mają więcej niż 25 lat.}

\begin{minted}{sql}
SELECT *
FROM Pracownicy
WHERE DATEDIFF(yy, DataUrodzenia, GETDATE()) > 25;
\end{minted}

\subsection{Dla każdego pracownika wyświetl imię, nazwisko, rok urodzenia, rok zatrudnienia oraz wiek, w którym został zatrudniony}

\begin{minted}{sql}
SELECT Imie, Nazwisko, YEAR(DataUrodzenia) AS RokUrodzenia,
YEAR(DataZatrudnienia) AS RokZatrudnienia,
DATEDIFF(yy, DataUrodzenia, DataZatrudnienia) AS WiekZatrudnienia
FROM Pracownicy;
\end{minted}

\subsection{Policz średni wiek wszystkich pracowników}

\begin{minted}{sql}
SELECT AVG(DATEDIFF(yy, DataUrodzenia, GETDATE())) AS SredniWiek
FROM Pracownicy;
\end{minted}
\subsection{Znajdź najpóźniejszą datę urodzenia spośród pracowników}
\begin{minted}{sql}
SELECT MAX(DataUrodzenia) AS NajpozniejszeUrodzenie
FROM Pracownicy;
\end{minted}

\subsection{Policz średnią wieku aktywnych pracowników w poszczególnych zespołach. Wyświetl te zespoły, w których średnia wieku jest większa niż 27 lat.}

\begin{minted}{sql}
SELECT Nazwa, AVG(DATEDIFF(yy, DataUrodzenia, GETDATE())) AS SredniaWieku
FROM Zespoly
INNER JOIN Pracownicy ON Pracownicy.NrZespolu = Zespoly.Id
WHERE Aktywny = 1
GROUP BY NrZespolu
HAVING AVG(DATEDIFF(yy, DataUrodzenia, GETDATE())) > 27;
\end{minted}
\end{document}
