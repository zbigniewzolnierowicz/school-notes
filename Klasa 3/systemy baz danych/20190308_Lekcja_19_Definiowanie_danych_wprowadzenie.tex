\documentclass[a4paper]{article}
\usepackage[T1]{fontenc}
\usepackage[polish]{babel}
\usepackage{minted}
\begin{document}

\title{Lekcja 19 --- Definiowanie danych --- wprowadzenie}
\author{Zbigniew Żołnierowicz}
\date{08.03.2019}

\maketitle

\section{Wprowadzenie}

Instrukcje języka DDL służą do manipulowania bazą danych i jej obiektami. Pozwalają one na:
\begin{itemize}
  \item Tworzenie nowych obiektów
  \item Modyfikowanie obiektów już istniejących
  \item usuwanie obiektów
\end{itemize}
Do podstawowych instrukcji języka DDL należą:
\begin{itemize}
  \item {\tt CREATE} --- służy do tworzenia nowych obiektów
  \item {\tt ALTER} --- służy do modyfikacji obiektów już istniejących
  \item {\tt DROP} --- służy do usunięcia obiektu już istniejącego
\end{itemize}
\section{\tt CREATE TABLE}

Przy definiowaniu tabeli należy podać jej nazwę, nazwy jej atrybutów oraz typ danych, które te atrybuty mogą przyjmować. Przykład tworzenia tabeli pokazano poniżej.

\begin{minted}{sql}
CREATE TABLE Osoba (
ID_Osoba int NOT NULL,
Nazwisko char(50) NOT NULL,
Imie char(20) NOT NULL,
Telefon int);
\end{minted}

\section{\tt ALTER TABLE}

Najczęściej polecenie {\tt ALTER TABLE} stosuje się do zmiany schematu relacji. Przykład zastosowania tego polecenia do dodania lub usunięcia kolumny z tabeli:

\begin{minted}{sql}
ALTER TABLE nazwa_tabeli ADD nazwa_kolumny typ_danych
\end{minted}

\begin{minted}{sql}
ALTER TABLE nazwa_tabeli DROP COLUMN nazwa_kolumny
\end{minted}

\section{\tt DROP TABLE}

W celu usunięcia istniejącej tabeli, należy zastosować polecenie {\tt DROP TABLE}. Usuwa ono również schemat tabeli.

Jego uproszczoną składnię częściową:
\begin{minted}{sql}
DROP TABLE nazwa_tabeli
\end{minted}
gdzie w nazwie podajemy nazwę obiektu, który chcemy usunąć. Przykładowo:
\begin{minted}{sql}
DROP TABLE Osoba;
\end{minted}

\section{Ograniczenie {\tt PRIMARY KEY}}

Najważniejszym ograniczeniem, jakie możemy przypisać do kolumny, jest klucz {\tt PRIMARY KEY} (pole musi mieć wartości niepuste i unikalne).

\begin{minted}{sql}
  CREATE TABLE Nowa (
  Kolumna1 int PRIMARY KEY,
  Kolumna2 int
  );
\end{minted}

\section{Ograniczenie {\tt UNIQUE}}

Podobnie jak klucz główny ograniczenie {\tt UNIQUE} nie pozwala na duplikowanie wartości. Umożliwia jednak wprowadzanie wartości NULL.\@

\begin{minted}{sql}
  CREATE TABLE Nowa (
  Kolumna1 int UNIQUE,
  Kolumna2 int
  );
\end{minted}

\section{Ograniczenia}

Specyfikacja ograniczenia nie musi pojawiać się bezpośrednio w definicji kolumny tabeli. Może ono zostac utworzone w dowolnym miejscu zapytania tworzącego tabelę, ale po definicji używanej do utworzenia ograniczenia kolumny.
W takim przypadku podajemy rodzaj więzów, a w nawiasie określamy kolumnę, której one dotyczą (nazwa ograniczenia nadawana jest przez system).

\begin{minted}{sql}
  CREATE TABLE Nowa (
  Kolumna1 int PRIMARY KEY,
  Kolumna2 int,
  );
\end{minted}

Tu system przypisuje nazwę ograniczeniu.

\begin{minted}{sql}
  CREATE TABLE Nowa (
  Kolumna1 int,
  Kolumna2 int,
  PRIMARY KEY(Kolumna1)
  );
\end{minted}

Tu my przypisujemy nazwę temu ograniczeniu.

\begin{minted}{sql}
  CREATE TABLE Nowa (
  Kolumna1 int,
  Kolumna2 int,
  CONSTRAINT pk PRIMARY KEY(Kolumna1)
  );
\end{minted}

W przypadku ograniczenia definiowanego poza definicją kolumny możliwe jest odwołanie się do większej liczby pól.

\begin{minted}{sql}
  CREATE TABLE Nowa (
  Kolumna1 int,
  Kolumna2 int,
  PRIMARY KEY(Kolumna1, Kolumna2)
  );
\end{minted}

\begin{minted}{sql}
  CREATE TABLE Nowa (
  Kolumna1 int,
  Kolumna2 int,
  CONSTRAINT pk PRIMARY KEY(Kolumna1, Kolumna2)
  );
\end{minted}

W przykładzie pokazany został wielokrotny klucz podstawowy określony przez parę kolumn. Oznacza to, że unikalna i niepusta jest para kolumn.
Należy pamiętać, że można odwoływać się do kolumn, które już zostały zdefiniowane i występują w poleceniu tworzącym tabelę przed miejscem określenia ograniczenia.

Takie same zasady dotyczą ograniczenia ukikalności tworzonego poza definicją kolumny.

Ponieważ ograniczenie {\tt UNIQUE} jest różne od {\tt PRIMARY KEY} tylko w zakresie wartości pustych, pierwsze z nich możemy uczynić równoważne drugiemu przez dodanie ograniczenia {\tt NOT NULL} w definicji właściwej kolumny lub kolumn.

\begin{minted}{sql}
  CREATE TABLE Nowa (
  Kolumna1 int,
  Kolumna2 int,
  PRIMARY KEY(Kolumna1)
  );
\end{minted}

\begin{minted}{sql}
  CREATE TABLE Nowa (
  Kolumna1 int NOT NULL,
  Kolumna2 int,
  CONSTRAINT un UNIQUE(Kolumna1)
  );
\end{minted}

\end{document}
