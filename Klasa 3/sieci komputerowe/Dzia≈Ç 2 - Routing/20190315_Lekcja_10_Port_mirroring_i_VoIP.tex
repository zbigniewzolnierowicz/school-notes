\documentclass{article}
\usepackage{polski}
\usepackage{indentfirst}
\usepackage[utf8]{inputenc}
\begin{document}
\title{Lekcja 10 --- Port mirroring i VoIP}
\author{Zbigniew Żołnierowicz}
\date{15.03.2019}
\maketitle
\section{Port mirroring}
\textbf{Port mirroring}, inaczej {\tt SPAN} \emph{(Switched Port Analyzer)},
to metoda monitorowania ruchu w sieci. Po włączeniu funkcji portu mirroring
przełącznik wysyła kopię wszystkich pakietów sieciowych widocznych na jednym porcie
(lub całej sieci {\tt VLAN}) do innego portu, w którym można przeanalizować pakiet.
\section{Zasada działania}
Przesyła pakiety z portu odbierającego \emph{(source)} do portu analizującego
\emph{(destination)}. Można ustawić, czy monitoruje się
wszystkie pakiety, tylko przychodzące \emph{(ingress)} lub
tylko wychodzące \emph{(egress)}.
\section{Pojęcie {\tt VoIP}}
{\tt VoIP} \emph{Voice over IP} to technologia, która umożliwia przesyłanie głosu przy pomocy
łączy dedykowanych lub internetowych sieci które wykorzystują protokoły {\tt IP},
równorzędnie nazywana jest telefonią internetową lub telefonią {\tt IP}.

W trakcie rozmowy przez {\tt IP} głos zostaje zamieniony na wersję cyfrową,
następnie skompresowany i podzielony na pakiety. Po tej zamianie komplet
pakietów jest przesyłany poprzez sieć internetową wraz z innymi informacjami.
U odbiorcy cały proces jest powielany jednak w odwrotnym kierunku. W wyniku
tych zabiegów otrzymujemy zwykły głos. Do zastosowania nadaje się każda sieć {\tt IP}.
Jedyny wymóg to komutacja pakietów.

Żeby połączenie {\tt VoIP} mogło być zrealizowane potrzebne są odpowiednie
urządzenia lub oprogramowanie.
\pagebreak
\section{Elementy {\tt VoIP}}
\subsection{Bramka systemu {\tt VoIP}}
Jest odpowiedzialna w trybie rzeczywistym za połączenie sieci IP do innych typów sieci.
Jej interfejs zapewnia skuteczną wymianę danych między różnymi formatami
transmisji danych i procedur telekomunikacyjnych. Przykładowe działanie bramki
to współdziałanie z centralą telefoniczną {\tt PBX}
lub zwykłą telefonią {\tt PSTN} lub {\tt ISDN}.
\subsection{Serwery, routery}
Stosowane są do tworzenia prywatnych sieci głosowych i nadzorowania w danej sieci transmisji
rozmów, kontroli dostępowego pasma, do routowania zgłoszeń, autoryzacji
użytkowników, a także przyjmowania i odrzucania zgłoszeń w strefie.
\subsection{Telefony}
Oprogramowanie zainstalowane na komputerze \emph{(softphone'y)},
lub dedykowane urządzenia działające bezpośrednio w sieciach IP.
Dzięki nim wykonujemy połączenia w sieci {\tt VoIP}.
\subsection{Protokoły, kodeki dla systemu}
Są to zestawy standardów kompresujących i dekompresujących dźwięk,
a także przesyłania pakietów i komunikację urządzeń w celu wymiany danych.
\section{Protokoły {\tt VoIP}}
{\tt VoIP} czyli telefonia internetowa wykorzystuje dwa typy protokołów:
\begin{description}
    \item[Protokół sygnalizacyjny] Jego zadanie to pośrednictwo w komunikacji urządzeń {\tt VoIP}.
    \item[Protokół transportujący] Jego zadaniem jest odpowiednia kontrola nad \\
    przesyłanymi pakietami głosowymi.
\end{description}
\pagebreak
\subsection{Protokoły IP}
\begin{itemize}
    \item {\tt SIP}
    \item {\tt LTP}
    \item {\tt H.323}
    \item {\tt MGCP}
    \item {\tt SCCP}
    \item {\tt IAX}
    \item {\tt Megaco}
    \item {\tt SIMPLE}
    \item {\tt RTP}
    \item {\tt STUN}
    \item {\tt ENUM}
    \item {\tt TRIP}
    \item {\tt T.37}
    \item {\tt T.38}
    \item {\tt COPS}
    \item {\tt PINT}
    \item {\tt SCTP}
\end{itemize}
\pagebreak
\subsubsection{\tt SIP}
Protokół {\tt SIP} \emph{(Session Initiation Protocol)}, jest protokołem sygnałowym,
działającym w aplikacjach. Służy on do ustalania adresów IP i numerów portów
wykorzystywanych w trakcie sesji multimedialnej.

{\tt SIP} to nie jest protokół transportowy (nie służy do transmisji danych),
a same pakiety danych nie są kierowane tą samą trasą co pakiety SIP.
Istnieje jednak mechanizm umożliwiający przesyłanie zdjęć, danych z wizytówek
lub stron internetowych łącznie z pakietami sygnalizacyjnymi.

{\tt SIP} posiada ograniczenia i kończy się jego rola na sygnalizacji bez kontroli
przesyłu danych. Zamysłem koncepcji {\tt SIP} jest prosta, przejrzysta warstwa
telekomunikacyjna, umożliwiająca współpracę jednostek internetowych oraz usług dodatkowych
zbudowanych nad tą warstwą.

Jedną z zalet protokołu {\tt SIP} jest fakt, iż jest on znacznie mniej skomplikowany
i przypomina protokoły {\tt HTTP}/{\tt SMTP}. Identyfikacja terminali końcowych odbywa się z
wykorzystaniem adresów podobnych do adresów e-mail: {\tt użytkownik@domena:port},
domyślnym portem jest 5060.
Dlatego też większość sprzętu {\tt VoIP} jest stworzona na podstawie standardu {\tt SIP};
podczas gdy starszy sprzęt {\tt VoIP} został stworzony ze standardem {\tt H.232}.
\subsubsection{\tt VoIP - H.323}
{\tt H.323} to standard opracowany i zaakceptowany przez {\tt ITU}, który oznacza zbiór
protokołóœ do prowadzenia multimedialnej komunikajci (głos, obraz, wiadomości tekstowe)
przez sieci komputerowe. {\tt H.323} jest stosunkowo przestarzałym protokołem i obecnie
jest zastępowany przez {\tt SIP} \emph{(ang. Session Initiation Protocol)}, protokół
inicjowania sesji.

Pierwszą wersję standardu przyjęto w 1997 roku. {\tt H.323} zaliczane jest do szeregu
standardów telekomunikacyjnych określanych jako {\tt H.23X}.

Typoszereg ten opisuje połączenia multimedialne w najróżniejszych sieciach, włączając w to
{\tt PSTN} i {\tt ISDN}.

W roku 1998 została zatwierdzona wersja tego standardu opisującą sposób tworzenia połączeń
multimedialnych w sieciach WAN.
\subsection{Protokół {\tt ITU}}
\begin{itemize}
    \item {\tt SS7}
    \item {\tt ISUP} --- standard związany z {\tt ITU}: {\tt P.1010}
\end{itemize}
\subsection{Inne}
\begin{itemize}
    \item {\tt OSP}
    \item {\tt PacketCable}
\end{itemize}
W większości rozwiązań w transporcie pakietów wykorzystywane są protokoły {\tt RTP}.
Są to protokoły transportowe trybu czasu rzeczywistego.
Dominujące protokoły to {\tt H.323} i {\tt SIP}.
\end{document}
