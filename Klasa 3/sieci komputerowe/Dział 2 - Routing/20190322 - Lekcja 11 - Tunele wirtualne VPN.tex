\documentclass[a4paper]{article}
\usepackage{booktabs}
\usepackage{indentfirst}
\usepackage{tgcursor}
\usepackage{inconsolata}
\usepackage[T1]{fontenc}
\usepackage[english,polish]{babel}
\usepackage[utf8]{inputenc}
\begin{document}
\title{
{\huge Lekcja 11} \\
{\large Tunele wirtualne VPN}
}
\author{Zbigniew Żołnierowicz}
\date{22.03.2019}
\maketitle
\section{\tt VPN}
{\tt VPN} \emph{(ang. Virtual Private Network)} polega na tworzeniu tuneli komunikacyjnych
w istniejącej sieci w celu stworzenia sieci wirtualnej. 
Wydzielana jest osobna wirtualna przestrzeń adresowa.
Tunel jest przezroczysty dla komunikujących się węzłów.
VPN pozwala utworzyć sieć prywatną za pomocą sieci publicznej (np.\@ internetu),
łączącą różne siedziby firmy lub organizacji, często odległe geograficznie.

W sieci publicznej należy się liczyć z potencjalnyumi naruszeniami poufności,
integralności i autentyczności transmitowanych danych. W celu realizacji bezpiecznego połączenia
tunel chroni się za pomocą kryptografii.
Dane przesyłane tunelem mogą też być kompresowane.
VPN działa w trzeciej warstwie (sieciowej) modelu ISO/OSI.

Na potrzeby wirtualnej sieci prywatnej tworzona jest osobna tablica routingu.
\section{Rodzaje architektury sieci {\tt VPN}}
\subsection{Tunel komputer-komputer}
Tunel komputer-komputer (ang. \emph{host-to-host} lub \emph{client-to-client}) to tunel, gdzie
końcami tunelu są pojedyncze stanowiska, wyposażone w odpowiednie oprogramowanie lub sprzęt do
szyfrowania transmisji pomiędzy nimi.
\subsection{Tunel sieć-sieć}
Tunel sieć-sieć (ang. \emph{net-to-net} lub \emph{site-to-site}) to tunel, gdzie końcami tunelu
są pojedyncze węzły międzysieciowe, np.\@ dedykowane urządzenia szyfrujące, routery brzegowe
z modułami kryptograficznymi. Szyfrowana może być cała transmisja wychodząca z sieci lokalnych
lub tylko wybrane usługi. Transmisja odbywająca się wewnątrz poszczególnych sieci nie jest
szyfrowana.
\subsection{Tunel komputer-sieć}
Tunel komputer-sieć (ang. \emph{host-to-net} lub \emph{client-to-site}) to tunel, gdzie jednym
z końców tunelu jest pojedyncze stanowisko, które uzyskuje dostęp do zasobów pewnej sieci
lokalnej, np.\@ korporacyjnej. Cała komunikacja lub wybrany ruch (wybrane usługi) poddawane są
szyfrowaniu. Jest to model typowy dla środowisk pracy zdalnej.
\section{\tt IPsec}
{\tt IPsec} oferuje:
\begin{itemize}
  \item Uniwersalny tunel wirtualny w warstwie sieciowej
  \item wzajemne uwierzytelnianie
  \item szyfrowanie datagramów IP
\end{itemize}
Jest wymaganą częścią {\tt IPv6} i ma 2 tryby pracy: transportowy i tunelowy.
\subsection{Tryb transportowy}
Pole danych niosące ramkę protokołu wyższej warstwy, np. {\tt TCP}, {\tt UDP}, {\tt ICMP}, itd.,
może zostać podpisane - do datagramu dodawany jest nagłówek {\tt AH}
(ang. \emph{authentication header}). Pole danych może też zostać
zaszyfrowane --- do datagramu dodawany jest wtedy nagłówek {\tt ESP}
(ang. \emph{Encapsulating Security Payload})
\subsection{Tryb tunelowy}
Cały oryginalny datagram IP, łącznie z nagłówkiem, jest podpisywany lub szyfrowany. Do datagramu
dodawane są odpowiednie nagłówki: {\tt AH}, {\tt ESP} i nowy nagłówek {\tt IP}.
\pagebreak
\subsection{Authentication Header {\tt (AH)}}
Authentication Header zapewnia integralność zawartości datagramu i uwierzytelnianie źródła
pochodzenia datagramu. Działa na wierzchu IP, używając numeru protokołu 51.

\begin{description}
  \item[Sequence Number] Monotonicznie zwiększana wartość, chroniąca przed atakiem powtórzeniowym.
  \item[Authentication Data] zawiera podpisany skrót zawartości.
\end{description}

Funkcja skrótu, oprócz pola danych, obejmuje też stałe pola nagłówka (zarówno w trybie transportowym,
jak i tunelowym). Ewentualna fragmentacja datagramu musi być wykonana wcześniej (podpisywany jest
każdy fragment oddzielnie.)

\begin{table}
  \begin{tabular}{@{}lll@{}}
    \toprule
    \multicolumn{1}{|l|}{Next Header (0 - 8)} & \multicolumn{1}{l|}{Payload Length (9 - 16)} & \multicolumn{1}{l|}{Reserved (16 - 31)} \\ \midrule
    \multicolumn{3}{|l|}{Security parameters Index} \\ \midrule
    \multicolumn{3}{|l|}{Sequence Number} \\ \midrule
    \multicolumn{3}{|l|}{Authentication Data} \\ \bottomrule
  \end{tabular}
  \caption{Authentication Header}
  \label{AH}
\end{table}

\subsection{Encapsulating Security Payload({\tt ESP})}

{\tt ESP} zapewnia integralność i poufność zawartości datagramu oraz uwierzytelnianie źródła pochodzenia datagramu.
Działa na wierzchu IP, używając numeru protokołu 50.
Można używać tylko samego szyfrowania \emph{(ang. encryption-only)} --- niezalecane lub uwierzytelniania \emph{(ang. authentication-only.)}
Nie chroni zawartości nagłówka IP (choć w trybie tunelowym chroni nagłówek oryginalnego datagramu.)
Payload data, pad length - wypełnienie dla szyfrów blokowych.

\begin{table}
  \begin{tabular}{|c|c|c|}
    \hline
    \multicolumn{3}{|c|}{Security Parameters Index} \\ \hline
    \multicolumn{3}{|c|}{Sequence Number} \\ \hline
    \multicolumn{3}{|c|}{Opaque Transform Data} \\ \cline{2-3} 
    (padding) & Pad Length & Next Header \\ \hline
  \multicolumn{3}{|c|}{\begin{tabular}[c]{@{}c@{}}Authentication Data\\ (opcjonalnie)\end{tabular}} \\ \hline
  \end{tabular}
  \caption{Encapsulating Security Payload}
  \label{ESP}
\end{table}
\pagebreak
\subsection{{\tt AH} i {\tt ESP}}

Możliwe jest połączenie {\tt AH} i {\tt ESP}. Przykładowo, najpierw szyfrowane są dane za pomocą {\tt ESP},
a następnie cały datagram jest podpisywany za pomocą {\tt AH}. Alternatywnie, najpierw wyznacza się nagłówek {\tt AH} i umieszcza się go w datagramie,

a następnie szyfruje w całości z użyciem {\tt ESP} (tuneluje).

\subsection{Security Parameters Index}

\emph{Security Parameters Index} ({\tt SPI}) identyfikuje parametry bezpieczeństwa.
W połączeniu z adresem {\tt IP} identyfikuje asocjację bezpieczeństwa {\tt SA} \emph{(ang. security association)} używaną dla danego datagramu.
{\tt SA} tworzy pewnego rodzaju kanał wirtualny.

\subsection{Asocjacja bezpieczeństwa}

Asocjacja bezpieczeństwa to zbiór parametrów charakteryzujących bezpieczną komunikację między nadawcą, a odbiorcą (kontekst):

\begin{itemize}
  \item algorytm {\tt AH}
  \item klucze {\tt AH}
  \item algorytm szyfrowania {\tt ESP}
  \item klucze szyfrujące {\tt ESP}
  \item dane inicjujące algorytm szyfrowania
  \item algorytm uwierzytelniania {\tt ESP}
  \item klucze algorytmu uwierzytelniania {\tt ESP}
  \item czas ważności kluczy
  \item czas ważności asocjacji
  \item adresy IP mogące współdzielić asocjację
  \item etykieta poziomu bezpieczeństwa
  \begin{itemize}
    \item poufne
    \item tajne
    \item ściśle tajne
  \end{itemize}
\end{itemize}

Parametry asocjacji bezpieczeństwa nie są przesyłane przez sieć --- przesyłany jest tylko numer {\tt SPI}.
Asocjacja bezpieczeństwa jest jednokierunkowa --- w każdym kierunku może być używany inny zestaw parametrów.

\subsection{Schemat działania stacji {\tt IPsec}}

\begin{enumerate}
  \item Sprawdź, czy i w jaki sposób wychodzący pakiet ma być zabezpieczony:
  \begin{itemize}
    \item sprawdź politykę bezpieczeństwa w {\tt SPD} \emph{(ang. Security Policy Database)}
    \item jeśli polityka bezpieczeństwa każe odrzucić pakiet, odrzuć pakiet
    \item jeśli pakiet nie musi być zabezpieczany, wyślij pakiet.
  \end{itemize}
  \item Ustal {\tt SA}, które powinno być zastosowane do pakietu:
  \begin{itemize}
    \item odszukaj {\tt SA} w bazie {\tt SAD} \emph{(ang. SA Database)} lub
    \item jeśli nie ma jeszcze odpowiedniego {\tt SA}, uzyskaj odpowiednie {\tt SA}.
  \end{itemize}
  \item Zabezpiecz pakiet, wykorzystując algorytmy, parametry i klucze zawarte w {\tt SA}:
  \begin{itemize}
    \item stwórz nagłówek {\tt AH}, {\tt ESP};
    \item jeśli tryb tunelowy, stwórz nowy nagłówek {\tt IP};
  \end{itemize}
  \item Wyślij pakiet.
\end{enumerate}

\subsection{Zarządzanie kluczami {\tt IPsec}}

{\tt IPsec} nie definiuje sposobów zarządzania i dystrybucji kluczy.
Klucze mogą być przypisane do użytkownika lub do komputera.

Sposoby dystrybucji kluczy:
\begin{itemize}
  \item Wyznaczenie wszystkich kluczy przez administratora (małej sieci lokalnej);
  \item Wykorzystanie istniejących systemów dystrybucji, np. Kerberosa;
  \item Specjalizowane protokoły dla serwerów kluczy (niezależne od {\tt IPsec}), np. {\tt SKIP} (\emph{Sun}), {\tt Photuris}, {\tt IKE} \emph{(Internet Key Exchange)};
  \item Integracja serwerów kluczy z usługami katalogowymi, np. {\tt DNSsec}, {\tt LDAP}
\end{itemize}

\subsubsection{Protokoły zarządzania kluczami {\tt IPsec}}

Służą do wzajemnego uwierzytelniania podmiotów nawiązujących asocjację {\tt IPsec} i do uzgadniania kluczy na potrzeby kanałów {\tt SA}.
Obie te funkcje realizowane są na podstawie skonfigurowanych na stałe danych uwierzytelniających:
\begin{itemize}
  \item Hasło wspólne dla pary stacji \emph{(ang. shared secret)}
  \item certyfikaty {\tt X.509}
  \item klucze {\tt PGP}
\end{itemize}

Niektóre implementacje ({\tt SKIP}, {\tt Photuris}) umożliwiają wyłącznie uwierzytelnianie na podstawie haseł.
Popularny protokół {\tt IKE} obsługuje natomiast wszystkie wyżej wymienione metowy i umożliwia jeszcze prywatne rozszerzenia.

\section{Protokół {\tt IKE} \emph{(Internet Key Exchange)}}

Obejmuje dwa składniki:
\begin{itemize}
  \item {\tt ISAKMP} \emph{(Internet Security Association and Key Management Protocol)} --- faktyczny protokół negocjacji parametrów {\tt IPsec};
  \item {\tt Oakley} --- kryptograficzny protokół wymiany kluczy za pomocą schematu Diffiego-Hellmana.
\end{itemize}

{\tt ISAKMP} stanowi trzon całości i z tego powodu nazwy tej używa się niekiedy zamiennie z {\tt IKE}.
Protokół {\tt ISAKMP} korzysta z portu {\tt UDP} (port 500).

Wymiana kluczy następuje dwuetapowo:

\begin{itemize}
  \item ustalenie tożsamożci komunikujących się węzłów i utworzenie bezpiecznego kanału (tzw. {\tt ISAKMP SA}), utrzymywanego przez cały czas trwania sesji;
  \item właściwa negocjacja parametrów asocjacji
\end{itemize}

Uwierzytelnianie może być realizowane na dwa sposoby:
\begin{itemize}
  \item Każda para węzłów ma ustalone wspólne hasło, które służy do obliczania kluczy metodą Diffiego-Hellmana; oznacza to konieczność konfigurowania haseł na wszystkich węzłach, co jest istotnym ograniczeniem i może okazać się zbyt pracochłonne w przypadku dużych sieci;
  \item Zastosowanie kluczy publicznych podpisanych przez nadrzędny urząd certyfikujący {\tt CA} (np.\@ certyfikatów {\tt X.509}); wolne od ograniczeń ręcznej definicji haseł.
\end{itemize}

Protokół {\tt ISAKMP} jest łatwo rozszerzalny, można zdefiniować własny zestaw szyfrów i własne mechanizmy uwierzytelniania.

\subsection{{\tt IKE} i {\tt PKI} \emph{(Public Key Infrastructure)}}

Protokół {\tt IKE} pozwala wykorzystać możliwości {\tt PKI}. Po nawiązaniu komunikacji, ale przed uzgodnieniem {\tt ISAKMP SA}, węzeł może zweryfikować autentyczność certyfikatu drugiej strony dzięki podpisowi {\tt CA}.

W skrajnym przypadku węzeł nie musi nic wiedzieć o innych węzłach, z którymi będzie się łączył, lub które będą się łączyć z nim.

Wymaga to jedynie lokalnego dostępu (zainstalowania w tym węźle) klucza publicznego urzędu {\tt CA} --- będzie to jeden i ten sam klucz na wszystkich węzłach. Znacznie ułatwia to realizację złożonych topologii.
\pagebreak
\subsection{\tt IKE}
Umożliwia automatyczną renegocjację kluczy kryptograficznych co określony interwał czasu.
W przypadku złamania bierzącego klucza dane zaszyfrowane poprzednimi kluczami nie są narażone.

Cecha ta, określana jako \emph{Perfect Forward Security}, chroni przed sytuacją, gdy atakujący zapisuje wszystkie przechwycone w przeszłości dane w nadziei, że kiedyś uda mu się zdobyć klucz do ich rozszyfrowania.
Implementacja jest obligowana, aby w przypadku renegocjacji klucza poprzedni klucz został usunięty z pamięci.
Wówczas włamywacz nie znajdzie go nawet w przypadku opanowania systemu operacyjnego węzła.
\section{{\tt PPTP} i {\tt MPTP}}
\section{\tt L2TP}
\section{\tt SSL}
\end{document}