\documentclass[a4paper]{article}
\usepackage{polski}
\usepackage[utf8]{inputenc}
\begin{document}
\title{Lekcja 13 --- Ideologie i doktryny polityczne}
\author{Zbigniew Żołnierowicz}
\date{14.03.2019}
\maketitle
\section{Czy doktryna i ideologia to to samo?}
\begin{description}
  \item[Ideologia] Zestaw idei
  \item[Doktryna] Wprowadzenie idei w życie
  \item[Doktryna Trumanna] Dotacje Stanów Zjednoczonych dla państw walczących z komunizmem
\end{description}
\section{Klasyfikacja doktryn politycznych}
\subsection{Ze względu na sposób wprowadzanych zmian}
\begin{description}
  \item[Ewolucjne (reformistyczne)] Powolne, legalne reformy.
  \item[Rewolucyjne] Szybki przewrót.
  \item[Konserwatywne] Bardzo mało lub brak zmian.
  \item[Reakcyjne] Zmiany, ale na to co było wcześniej.
\end{description}
\subsection{Ze względu na światopogląd}
Ze względu na światopogląd możemy doktryny podzielić na lewicę, prawicę i centrum.
\begin{description}
  \item[Lewica] Mniej przywiązani do tradycji, bardziej otwarci na innych ludzi, otwarci na zmiany
    \begin{itemize}
      \item Socjalizm
    \end{itemize}
  \item[Centrum] Wszystko pomiędzy.
    \begin{itemize}
      \item Liberalizm
    \end{itemize}
  \item[Prawica] Tradycyjny, duże związanie do religii i tradycji, konserwatywny, mówią więcej o narodzie.
    \begin{itemize}
      \item Neofaszyzm
      \item Konserwatyzm
    \end{itemize}
\end{description}
\end{document}
