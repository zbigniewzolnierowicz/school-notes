\documentclass{article}
\usepackage[T1]{fontenc}
\usepackage[english,polish]{babel}
\usepackage[utf8]{inputenc}
\let\oldquote\quote
\let\endoldquote\endquote
\renewenvironment{quote}[2][]
  {\if\relax\detokenize{#1}\relax
     \def\quoteauthor{#2}%
   \else
     \def\quoteauthor{#2~---~#1}%
   \fi
   \oldquote}
  {\par\nobreak\smallskip\hfill(\quoteauthor)%
   \endoldquote\addvspace{\bigskipamount}}
\begin{document}
\title{
    {\huge Lekcja 14} \\
    {\large Charakerystyka współczesnych doktryn politycznych}
}
\author{Zbigniew Żołnierowicz}
\date{21.03.2019}
\maketitle
\section{Konserwatyzm}
\subsection{Człowiek}
Człowiek ma dążyć do zachowania istniejącego ładu, ustalonego przez \emph{religię, tradycję, historię}.
Bóg jest stwórcą świata, a człowiek - stworzony przez boga - ma stać na straży tego porządku.
\subsection{Religia}
Religia źródłem systemu wartości.
\subsection{Rodzina}
Rodzina to pierwotny wzór społeczeństwa hierarchicznego.
\subsection{Społeczeństwo}
Społeczeństwo to organizm, którego narządy są zróżnicowane i każdy odgrywa swoją rolę.
Powinno być różnorodne i hierarchiczne (duża rola elit).
\subsection{Własność}
Własność prywatna.
\subsection{Ład społeczny}
Sprzeciw wobec wszelkim rewolucjom, gwałtowne zmiany są wbrew naturze człowieka.
\section{Katolicka nauka społeczna}
\subsection{Człowiek}
Człowiek odkrywa Boga w człowieku i człowieka w Bogu. Istnieje życie doczesne i życie wieczne.
Prawa człowieka wynikają z godności osoby ludzkiej i przysługują człowiekowi bezwzględnie.
\subsection{Zasada solidarności i solidaryzmu społecznego}
Zasada solidarności i solidaryzmu społecznego, czyli uporządkowanej zgody w dążeniu do wspólnego dobra.
\begin{quote}{Leon XIII, encyklika \emph{Rerum novarum}.}
    "Ani kapitał bez pracy, ani bez kapitału praca nie może istnieć."
\end{quote}
\subsection{Własność}
Własność prywatna, ograniczona rola państwa w gospodarce.
\subsection{Ład społeczny i pokój}
Ład społeczny i pokój zależą od zachowania porządku ustawionego przez Boga.
\section{Liberalizm}
\subsection{Człowiek}
Człowiek jako jednostka jest wartością nadrzędną.
\emph{Racjonalistyczna i indywidualistyczna koncepcja człowieka.}
\subsection{Wolność}
Jest podstawową wartością, rozumianą jako swoboda działania.
Nierozerwalnie łączy się ona z odpowiedzialnością za podejmowane granice.
\begin{quote}{Alexis de Tocqueville}
    Granice wolności jednostki wyznacza wolność drugiego człowieka.
\end{quote}
\subsection{Równość}
Równość jest rozumiana jako równość wobec praw natury. Ludzie rodzą się równi i
mają równe prawa do korzystania z wolności; jedni z niej korzystają, inni nie.
Jest to sprawiedliwe. Nie istnieje coś takiego jak ``sprawiedliwość społeczna''.
\begin{quote}{Autor nieznany}
    "Mają ci, którzy chcą mieć; ci, którzy się o to troszczą."
\end{quote}
\subsection{Społeczeństwo}
Społeczeństwo jest sumą jednostek lub grup społecznych o przeciwstawnych interesach.
\subsection{Państwo}
Państwo ma pełnić rolę ``nocnego stróża''; Programy socjalne są przejawem nierównego traktowania
obywateli przez państwo, które nagradza nieudaczników.
\begin{quote}{Autor nieznany}
    Co lepiej dać biedakowi: rybę czy wędkę?
    Odpowiedź liberała: ``Każdy z nas dysponuje wędką, dlatego państwo nie powinno rozdawać ryb tym,
    którzy nie potrafią łowić lub po prostu nie chcą iść na ryby.''
\end{quote}
\section{Socjaldemokracja}
\subsection{Człowiek}
Człowiek jest głęboko osadzony w społeczeństwie, pojmowany nieco idealistycznie. Gdy ponosi klęski,
winy upatruje się w złym porządku społecznym.

\emph{Brak sprawiedliwości społecznej}
\subsection{Społeczeństwo}
Społeczeństwo złożone z klas o przeciwstawnych interesach, konieczne jest zapewnienie praw socjalnych
potrzebującym.
\subsection{Równość}
Od XX w.\@ nacisk na równość szans, równy start życiowy (np.\@ powszechny dostęp do nauki).
\subsection{Wolność}
Wolność jest pojęciem problematycznym; jeśli własność prywatna pociąga za sobą przymus ekonomiczny
to wolność jest pustym frazesem; współcześnie socjaldemokraci są bliscy pojmowaniu wolności przez
liberałów.
\subsection{Państwo}
Możliwy i konieczny interwencjonizm państwowy.
\subsection{Własność}
Początkowo ideałem była własność społeczna, dziś uznaje się wyższość własności prywatnej. Ideałem
byłoby, gdyby produkcja produkcja opierała isę na zasadach kapitalizmu, a podział dóbr - socjalizmu.
\end{document}