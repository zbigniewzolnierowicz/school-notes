\documentclass[a4paper]{article}
\usepackage{polski}
\usepackage[utf8]{inputenc}
\usepackage{datetime}
\usepackage{tikz}
\begin{document}
\title{
        {\huge Lekcja 33} \\
        {\large Na czym polega ``budowanie siebie'' wg.\@ Leopolda Staffa?}
}
\author{Zbigniew Żołnierowicz}
\date{01.04.2019}
\maketitle
\section{Uzupełnij schemat dotyczący podmiotu lirycznego wiersza \emph{Kowal}.}
\large{Podmiot liryczny wiersza \emph{Kowal} --- Kowal}
\normalsize
\subsection{Konflikt 1}
\begin{quote}
        Serce hurtowne, mężne, serce, dumne, silne
\end{quote}
\paragraph*{Działania}
Kowal kuje dla siebie serce mężne, bez skazy.
\paragraph*{Efekty}
Wzrasta hart ducha.
\paragraph*{Nazwa filozofii, do której odnosi się postawa} Aktywizm, Nietzschenizm.
\subsection{Konflikt 2}
\begin{quote}
        serce [\dots] własną słabością przeklęte, \\
        Rysą chorej niemocy skażone
\end{quote}
\paragraph*{Działania}
Kowal kuje serce słabe.
\paragraph*{Efekty}
Woli umrzeć, niż mieć słabe serce.
\paragraph*{Nazwa filozofii, do której odnosi się postawa}
Schopenhaueryzm.
\section{Jakie nawiązania do starozytnych tekstów kultury można odnaleźć w utworze Leopolda Staffa? Czemu one służą?}
Sam kowal przypomina postać Hefajstosa.

Pojawiają się cyklopy i wulkan.

Celem tych odniesień ma za zadanie nadania wagi temu tekstu.
\section{Na czym polega odmienność \emph{Kowala} Leopolda Staffa od innych --- poznanych przez Ciebie -- wierszy młodopolskich?}
Wiersze młodopolskie to ubolewanie nad własną sytuacją, a \emph{Kowal} propaguje proaktywne podejście do polepszania siebie.
\section{Wnioski}
Kowal wybiera postawę nietzcheańską. Kowal jest w pewnym sensie nadczłowiekiem.
\section{W jaki sposób autor eksponuje kolor i muzyczność utworu? Podaj przykłady}
\begin{itemize}
        \item Budowa klamrowa
        \begin{quote}
                O szyby \emph{deszcz dzwoni, deszcz dzwoni} jesienny
        \end{quote}
        \item Zastosowanie wyrazów onomatopeicznych i kojarzonych z instrumentami
        \begin{quote}
                I \emph{pluszcze} jednaki, miarowy, niezmienny
        \end{quote}
        \begin{quote}
                O szyby deszcz \emph{dzwoni}, deszcz \emph{dzwoni} jesienny
        \end{quote}
        \begin{quote}
                \emph{Jęk} szklany\dots \emph{płacz} szklany\dots a szyby w mgle mokną
        \end{quote}
        \item Epitety opisujące rytm i kolor
        \begin{itemize}
                \item jednaki, miarowy, niezmienny
                \item jesienny
                \item szarą i mglistą
                \item chmurny dzień słotny
        \end{itemize}
        \pagebreak
        \item Anafora
        \begin{quote}
                \emph{Ktoś} dziś mnie opuścił w ten chmurny dzień słotny\dots \\
                \emph{Kto?} Nie wiem\dots Ktoś odszedł i jestem samotny\dots \\
                \emph{Ktoś} umarł\dots Kto? Próżno w pamięci swej grzebię\dots \\
                \emph{Ktoś} drogi\dots wszak byłem na jakimś pogrzebie\dots
        \end{quote}
        \item Powtarzanie zwrotki, nadawanie jej stylu refrenu
        \begin{quote}
                To w szyby deszcz dzwoni, deszcz dzwoni jesienny \\
                I pluszcze jednaki, miarowy, niezmienny, \\
                Dżdżu krople padają i tłuką w me okno\dots \\
                Jęk szklany\dots płacz szklany\dots a szyby w mgle mokną \\
                I światła szarego blask sączy się senny\dots \\
                O szyby deszcz dzwoni, deszcz dzwoni jesienny\dots
        \end{quote}
        \item Liczne rymy
        \item Podział na zwrotki
\end{itemize}
\subsection{Dlaczego możemy powiedzieć, że Staff prezentuje w utworze postawę dekadencką?}
Wiersz jest wypełniony smutkiem i wydarzeniami, które są dość przerażające lub zasmucające (Płonąca wioska, szatan zmieniający ogród w pustelnię, powtarzająca się zwrotka o deszczu).
\end{document}