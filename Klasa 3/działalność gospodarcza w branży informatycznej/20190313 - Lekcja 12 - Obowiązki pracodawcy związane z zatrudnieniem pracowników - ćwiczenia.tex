\documentclass[a4paper]{article}
\usepackage{polski}
\usepackage[utf8]{inputenc}
\begin{document}
\title{Lekcja 12 --- Obowiązki pracodawcy związane z zatrudnieniem pracowników --- ćwiczenia}
\author{Zbigniew Żołnierowicz}
\date{13.03.2019}
\maketitle
\section{Jakie są według ciebie największe zalety outsourcingu?}

\begin{itemize}
  \item Niskie koszta
  \item Wysoka jakość --- firma do której outsourcujesz jest wyspecjalizowana w danej dziedzinie.
  \item Pełne skupienie --- firma, którą zatrudniasz jest w 100\% zadedykowana danej pracy, więc twoja firma może się skupić na innych sprawach.
\end{itemize}

\section{W jakich sytuacjach jako pracodawca korzystałbyśz pracy tymczasowej?}
\section{Wyjaśnij różnicę pomiędzy umową o dzieło a umową-zleceniem, zwracając szczególną uwagę na to, że jedna z nich jest jest umową starannego działania, a druga umową rezultatu.}
\section{Wyjaśnij działanie zasady swobody umów i zasady nadrzędności prawa pracy. Która z tych zasad jest stosowana w przypadku umów o pracę?}
\section{Jaką umowę powinien podpisać z pracownikiem pracodawca, jeżeli chce, aby praca była wykonywana we wskazanym przez niego miejscu i czasie oraz pod jego kierownictwem?}
\section{Dlaczego kwota wynagrodzenia, jaką pracownik otrzymuje na konto z tytułu pracy wykonywanej na rzecz pracodawcy, jest inna niż kwota wynagrodzenia w umowie o pracę?}
\section{Czym różni się instruktaż stanowiskowy od instruktaża ogólnego?}

Instruktaż stanowiskowy przeprowadzany jest dla osób które zostają zatrudnieni na określone stanowisko. Instruktaż ten jest także specyficzny dla określonego stanowiska, podczas gdy instruktaż ogólny jest udzielany każdemu, m.in.\@ praktykantom i stażystom i dotyczy ogólnych zasad BHP.\@

\section{Z jakich części składają się akta osobowe pracownika?}
\section{Jakie obowiązki ma pracodawca w związku z pełnieniem roli płatnika składek na ubezpieczenia społeczne i ubezpieczenie zdrowotne?}
\section{Jakie obowiązki ma pracodawca w związku z pełnieniem roli płatnika podatku dochodowego od osób fizycznych od wypłacanych pracownikom wynagrodzeń?}
\section{Jakie uprawnienia związane z rodzicielstwem przysługują pracownikom?}
\section{Czy pracodawca może uzależniać wydanie pracownikowi świadectwa pracy od spełnienia przez pracownika jakichkolwiek warunków?}
\end{document}